\begin{center} \centering \label{ucCreateTopic}
	\begin{longtable}{|p{4.6cm}|p{9.4cm}|}  %% Longtable = forsætter på næste side
	\hline
		\multicolumn{2}{|l|}{\textbf{UC1: Create Topic}} \\\hline
		\endfirsthead
		
		\multicolumn{2}{l}{...fortsat fra forrige side} \\ \hline %% Til LONGTABLE
		\multicolumn{2}{|l|}{\textbf{UC1: Create Topic}} \\\hline
		\endhead	
		
		\textbf{Mål}						&Bruger har oprettet et topic
		\\\hline
		\textbf{Initialisering}			&Ingen (?bruger logget ind?)
		\\\hline
		\textbf{Aktører og Stakeholders}	&Bruger (evt. system administrator (os))
		\\\hline 
		\textbf{Referencer}				&%SKRIV
		\\\hline
		\textbf{AASH}					&Én hændelse per bruger
		\\\hline
		\textbf{Efterfølgende tilstand}	&?
		\\\hline
		\textbf{Hovedforløb}					
			&\begin{enumerate}
				% Tilføj punkter for forløbet med items
				%
				% \item blablabla
				%
				% \item \label{ucPunktLabel} blablablabla
				%
				% \item \label{ucPunktLabel} blabla med undtagelse
				%
				% \textbf{[Undtagelse \ref{ucPunktLabel}.a]} \newline
				% 	ALERT!!
				
				\item Bruger trykker på "Create Topic"
				\item Brugergrænsefladen opdaterer, således at de forskellige konfigurationsmuligheder bliver præsenteret
				\item Bruger indtaster navn på topic
				\item Bruger indtaster søgbare tags på topic
				\item Bruger indtaster det første spørgsmål
				\item Bruger indtaster svarmuligheder og angiver det korrekte svar
				\item Bruger kan tilføje flere spørgsmål ved at trykke "Create new question", hvormed punkt ??(spørgsmål) gentages
				
			\end{enumerate}\\\hline
		\textbf{Undtagelser}
			&			
			% Undtagelser laves som små enumeraters. Brug syntaksen her under, så punkter automatisk opdateres
			%\begin{enumerate}[label=\ref{ucPunktLabel}.a]
			%	\item Gå til UC1.\ref{ucPunktLabelSomSkalTil}
			%\end{enumerate}																
			\\\hline
	\end{longtable} 
\end{center}