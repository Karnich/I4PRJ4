\begin{center} \centering \label{ucFindQuiz}
	\begin{longtable}{|p{4.6cm}|p{9.4cm}|}  %% Longtable = forsætter på næste side
	\hline
		\multicolumn{2}{|l|}{\textbf{UC2: Find Quiz}} \\\hline
		\endfirsthead
		
		\multicolumn{2}{l}{...fortsat fra forrige side} \\ \hline %% Til LONGTABLE
		\multicolumn{2}{|l|}{\textbf{UC2: Find Quiz}} \\\hline
		\endhead	
		
		\textbf{Mål}					& Bruger har fundet den ønskede Quiz.
		\\\hline
		\textbf{Initialisering}			& Bruger aktiverer søgefeltet på brugergrænsefladen.
		\\\hline
		\textbf{Aktører og Stakeholders}	& Bruger, Database server, Web server.
		\\\hline 
		\textbf{Referencer}				& % evt. login i senere iteration
		\\\hline
		\textbf{AASH}					& 1 pr. session.
		\\\hline
		\textbf{Efterfølgende tilstand}	& Bruger har markeret den ønskede Quiz.
		\\\hline
		\textbf{Hovedforløb}					
			&\begin{enumerate}
				\item \label{ucFindQuizSearch} Bruger indtaster navnet på de tags, som den ønskede Quiz har tilknyttet, og trykker på SØG knappen på brugergrænsefladen. 
				\textbf{Undtagelse \ref{ucFindQuizSearch}.a}
				\item \label{ucFindQuizNoResults} Brugeren vises en række quizzer, som har tilknyttet de tags Bruger indtastede.
				\textbf{Undtagelse \ref{ucFindQuizNoResults}.a}
				%
				\item Bruger markerer den ønskede quiz.
				% \textbf{[Undtagelse \ref{ucPunktLabel}.a]} \newline
				% 	ALERT!!
				
			\end{enumerate}\\\hline
		\textbf{Undtagelser}
			&\begin{enumerate} [label=\ref{ucFindQuizSearch}.a]
				\item Bruger fjerner fokus fra søgefeltet.
				\subitem Use casen afsluttes.
			\end{enumerate}
			\begin{enumerate} [label=\ref{ucFindQuizNoResults}.a]
				\item Søgningen fandt ingen quizzer med de angivne tags.
				\subitem Gå til UC2 punkt \ref{ucFindQuizSearch}.
			\end{enumerate}
						
			% Undtagelser laves som små enumeraters. Brug syntaksen her under, så punkter automatisk opdateres
			%\begin{enumerate}[label=\ref{ucPunktLabel}.a]
			%	\item Gå til UC3.\ref{ucPunktLabelSomSkalTil}
			%\end{enumerate}																
			\\\hline
	\end{longtable} 
\end{center}