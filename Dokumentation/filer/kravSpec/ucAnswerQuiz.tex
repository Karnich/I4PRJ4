\begin{center} \centering \label{ucAnswerQuiz}
	\begin{longtable}{|p{4.6cm}|p{9.4cm}|}  %% Longtable = forsætter på næste side
		\hline
		\multicolumn{2}{|l|}{\textbf{UC3: Answer Quiz}} \\\hline
		\endfirsthead
		
		\multicolumn{2}{l}{...fortsat fra forrige side} \\ \hline %% Til LONGTABLE
		\multicolumn{2}{|l|}{\textbf{UC3: Answer Quiz}} \\\hline
		\endhead	
		
		\textbf{Mål}						&Bruger vil besvare en Quiz.
		\\\hline
		\textbf{Initialisering}			&Bruger har markeret en Quiz.
		\\\hline
		\textbf{Aktører og Stakeholders}	&Bruger, Database server og Web server.
		\\\hline 
		\textbf{Referencer}				&
		\\\hline
		\textbf{AASH}					&1 pr. session.
		\\\hline
		\textbf{Efterfølgende tilstand}	&Bruger har besvaret alle spørgsmål i en Quiz.
		\\\hline
		\textbf{Hovedforløb}					
			&\begin{enumerate}
			\item Bruger vælger ''Take Quiz'' på brugergrænsefladen.
			\item\label{ucAnswerQuizPresent} Bruger præsenteres for spørgsmål.
			\item Bruger vælger en af svarmulighederne. 
			\item Brugergrænsefladen viser det rigtige svar.

			\item\label{ucAnswerQuizEnd} Næste spørgsmål præsenteres for bruger \newline
			[Punkt \ref{ucAnswerQuizPresent} - \ref{ucAnswerQuizEnd} gentages indtil der ikke er flere spørgsmål i Quizzen]
			\item Bruger præsenteres for statistik der viser hvor mange spørgsmål der er besvaret korrekt.
			\item Bruger vælger at afslutte hvorved Quiz forlades


			\end{enumerate}\\\hline
		\textbf{Undtagelser}
			&			
			% Undtagelser laves som små enumeraters. Brug syntaksen her under, så punkter automatisk opdateres
			%\begin{enumerate}[label=\ref{ucPunktLabel}.a]
			%	\item Gå til UC3.\ref{ucPunktLabelSomSkalTil}
			%\end{enumerate}																
			\\\hline
	\end{longtable} 
\end{center}