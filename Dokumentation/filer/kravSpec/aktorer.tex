\section{Aktører}

Til applikationen Quiz Creator identificeres en række use cases samt aktører. Disses ses på figur \ref{fig:ucDiagram}, hvor primæraktører ses til venstre og sekundære aktører ses til højre. Følgende use case-diagrammet findes en beskrivelse over aktører, og derefter en gennemgang af hver use case. 


\figur{0.8}{kravspec/UCdiagramFinal3}{Use case diagram med aktører}{fig:ucDiagram}


\subsection{Aktør-beskrivelse}
Her følger en kort beskrivelse af de enkelte aktører vist i use case diagrammet på figur \ref{fig:ucDiagram}.

\begin{tabular}{|p{2cm}|p{12cm}|}

\hline 
\textbf{Navn} & Bruger \\ 
\hline 
\textbf{Type} & Primær aktør \\ 
\hline 
\textbf{Beskrivelse} & Bruger initierer alle use cases. Det er muligt at oprette en tilbagevendende bruger, eller anvende en midlertidig bruger.\\ 
\hline 

\end{tabular} 

\begin{tabular}{|p{2cm}|p{12cm}|}

\hline 
\textbf{Navn} & Databaseserver \\ 
\hline 
\textbf{Type} & Sekundær aktør \\ 
\hline 
\textbf{Beskrivelse} & En server der kan tilgås fra webserveren og som gemmer og håndterer data i databaser.\\ 
\hline 

\end{tabular} 

\begin{tabular}{|p{2cm}|p{12cm}|}

\hline 
\textbf{Navn} & Webserver \\ 
\hline 
\textbf{Type} & Sekundær aktør \\ 
\hline 
\textbf{Beskrivelse} & En server der opbevarer og fremsender web applikationens data på internettet. Den udfører desuden kompileret kode i programmets backend og tilgår databaseserveren. \\ 
\hline 

\end{tabular} 