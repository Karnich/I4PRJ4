\section{Aktører}

Til applikationen QuizCreator indentificeres en række use cases samt aktører. Disses ses i \ref{fig:ucDiagram}.
\figur{1.0}{kravspec/UC-diagram}{Use Case-diagram}{fig:ucDiagram}, hvor primær aktører ses til venstre og sekundære aktører ses til højre. Følgende Use case diagrammet findes en beskrivelse over aktører, og derefter en gennemgang af hver use case. 

\subsection{Aktør-beskrivelse}

\begin{tabular}{|p{2cm}|p{12cm}|}

\hline 
\textbf{Navn} & Bruger \\ 
\hline 
\textbf{Type} & Primær aktør \\ 
\hline 
\textbf{Beskrivelse} & Bruger initierer i alle tre use cases. Bruger kan oprette en Quiz, finder en Quiz og besvare en Quiz. Bruger identificeres ikke individuelt, men alle brugere af applikationen vil gå indenfor brugeren "Gæst"\\ 
\hline 

\end{tabular} 

\begin{tabular}{|p{2cm}|p{12cm}|}

\hline 
\textbf{Navn} & Database server \\ 
\hline 
\textbf{Type} & Sekundær aktør \\ 
\hline 
\textbf{Beskrivelse} & En ekstern server, som gemmer og håndterer databaser.\\ 
\hline 

\end{tabular} 

\begin{tabular}{|p{2cm}|p{12cm}|}

\hline 
\textbf{Navn} & Webserver \\ 
\hline 
\textbf{Type} & sekundær aktør \\ 
\hline 
\textbf{Beskrivelse} & Webserveren opbevarer og udleverer hjemmesidens data på internettet.\\ 
\hline 

\end{tabular} 