% Tabel header

\uchead
	{1}
	{Create quiz}
	{Bruger vil oprette en Quiz.}
	{Bruger trykker på ''Create Quiz'' på brugergrænsefladen.}
	{Bruger, Databaseserver og Webserver.}
	{}
	{1 pr. session.}
	{Bruger har oprettet en Quiz.}

\item Webserveren præsenterer de forskellige konfigurationsmuligheder, som Bruger har til rådighed ved ''Create Quiz''.

\item \label{ucCreateQuizName} Bruger indtaster navn på Quiz, og dette tilføjes automatisk som et søgbart tag til Quizzen.

\item Bruger indtaster søgbare tags på Quiz.

\item \label{ucCreateQuizQst} Bruger indtaster spørgsmål.

\item Bruger kan vælge at tilføje et billede til det pågældende spørgsmål ved at trykke på ''Gennemse...'' og vælge det ønskede billede.

\item \label{ucCreateQuizAns} Bruger indtaster svarmuligheder, angiver det/de korrekte svar, og tildeler point til de ønskede svarmuligheder.

\item Bruger kan tilføje flere svarmuligheder ved at trykke ''Add new answer'', hvormed punkt \ref{ucCreateQuizAns} gentages.

\item Bruger kan tilføje flere spørgsmål ved at trykke ''Add new question'', hvormed der hoppes til punkt \ref{ucCreateQuizQst}.

\item \label{ucCreateQuizDone} Bruger trykker på ''Create Quiz'' knappen.
\textbf{Undtagelse \ref{ucCreateQuizDone}.a}

\item \label{ucCreateQuizDB} Webserveren gemmer quizzen på databaseserveren og UC afsluttes.

\ucdescriptionend

\ucextension
	{De indtastede oplysninger om Quizzen er ikke i overensstemmelse med de ikke-funktionelle krav (se afsnit \ref{sec:nonFunctional}, side \pageref{sec:nonFunctional}).}
	{Gå til UC1 punkt \ref{ucCreateQuizName}}
	{\ref{ucCreateQuizDone}.a}
		
\ucfoot{CreateQuiz}