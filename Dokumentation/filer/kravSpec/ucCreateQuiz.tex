\begin{center} \centering \label{ucCreateQuiz}
	\begin{longtable}{|p{4.6cm}|p{9.4cm}|}  %% Longtable = forsætter på næste side
	\hline
		\multicolumn{2}{|l|}{\textbf{UC1: Create Quiz}} \\\hline
		\endfirsthead
		
		\multicolumn{2}{l}{...fortsat fra forrige side} \\ \hline %% Til LONGTABLE
		\multicolumn{2}{|l|}{\textbf{UC1: Create Quiz}} \\\hline
		\endhead	
		
		\textbf{Mål}						&Bruger vil oprette en Quiz.
		\\\hline
		\textbf{Initialisering}			& Bruger trykker på ''Create Quiz'' på brugergrænsefladen.
		\\\hline
		\textbf{Aktører og Stakeholders}	&Bruger, Database server og Webserver.
		\\\hline 
		\textbf{Referencer}				&%SKRIV
		\\\hline
		\textbf{AASH}					&1 pr. session.
		\\\hline
		\textbf{Efterfølgende tilstand}	&Bruger har oprettet en Quiz.
		\\\hline
		\textbf{Hovedforløb}					
			&\begin{enumerate}
				\item Brugergrænsefladen viser de forskellige konfigurationsmuligheder, som Bruger har til rådighed ved ''Create Quiz''.
				\item \label{ucCreateQuizName} Bruger indtaster navn på Quiz, og dette tilføjes automatisk som et søgbart tag til Quizzen.
				\item Bruger indtaster søgbare tags på Quiz.
				\item \label{ucCreateQuizQst} Bruger indtaster det første spørgsmål.
				\item Bruger indtaster svarmuligheder og angiver det korrekte svar.
				\item Bruger kan tilføje flere spørgsmål ved at trykke "Create new question", hvormed der hoppes til punkt\ref{ucCreateQuizQst} i use casen.
				\item \label{ucCreateQuizDone} Bruger trykker på ''Create'' knappen og UC afsluttes.
				\textbf{Undtagelse \ref{ucCreateQuizDone}.a}
				\textbf{Undtagelse \ref{ucCreateQuizDone}.b}
				
			\end{enumerate}\\\hline
		\textbf{Undtagelser}
			&\begin{enumerate} [label=\ref{ucCreateQuizDone}.a]
				\item Bruger vælger i løbet af use casens forløb at trykke ''Discard Quiz''.
				\subitem Quizzen oprettes ikke, og use casen afsluttes.
			\end{enumerate}
			\begin{enumerate} [label=\ref{ucCreateQuizDone}.b]
				\item De indtastede oplysninger om Quizzen er ikke i overensstemmelse med de ikke-funktionelle krav (se afsnit \ref{sec:nonFunctional} side \pageref{sec:nonFunctional} ).
				\subitem Gå til UC1 punkt \ref{ucCreateQuizName}.
			\end{enumerate}	
			% Undtagelser laves som små enumeraters. Brug syntaksen her under, så punkter automatisk opdateres
			%\begin{enumerate}[label=\ref{ucPunktLabel}.a]
			%	\item Gå til UC1.\ref{ucPunktLabelSomSkalTil}
			%\end{enumerate}																
			\\\hline
	\end{longtable} 
\end{center}