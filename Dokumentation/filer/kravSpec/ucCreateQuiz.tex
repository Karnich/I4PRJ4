\begin{center} \centering \label{ucCreateQuiz}
	\begin{longtable}{|p{4.6cm}|p{9.4cm}|}  %% Longtable = forsætter på næste side
	\hline
		\multicolumn{2}{|l|}{\textbf{UC1: Create Quiz}} \\\hline
		\endfirsthead
		
		\multicolumn{2}{l}{...fortsat fra forrige side} \\ \hline %% Til LONGTABLE
		\multicolumn{2}{|l|}{\textbf{UC1: Create Quiz}} \\\hline
		\endhead	
		
		\textbf{Mål}						&Bruger vil oprette en quiz
		\\\hline
		\textbf{Initialisering}			&Ingen
		\\\hline
		\textbf{Aktører og Stakeholders}	&Bruger, Database server og Web server
		\\\hline 
		\textbf{Referencer}				&%SKRIV
		\\\hline
		\textbf{AASH}					&1 pr. session
		\\\hline
		\textbf{Efterfølgende tilstand}	&Bruger har oprettet en quiz
		\\\hline
		\textbf{Hovedforløb}					
			&\begin{enumerate}
				\item Bruger trykker på ''Create Quiz''
				\item Brugergrænsefladen opdaterer, således at de forskellige konfigurationsmuligheder bliver præsenteret
				\item Bruger indtaster navn på quiz
				\item Bruger indtaster søgbare tags på quiz
				\item \label{ucPunktLabelSpm} Bruger indtaster det første spørgsmål
				\item Bruger indtaster svarmuligheder og angiver det korrekte svar
				\item Bruger kan tilføje flere spørgsmål ved at trykke "Create new question", hvormed \ref{ucPunktLabelSpm} gentages
				\item Bruger trykker på ''Create'' knappen og UC afsluttes
				
			\end{enumerate}\\\hline
		\textbf{Undtagelser}
			&			
			% Undtagelser laves som små enumeraters. Brug syntaksen her under, så punkter automatisk opdateres
			%\begin{enumerate}[label=\ref{ucPunktLabel}.a]
			%	\item Gå til UC1.\ref{ucPunktLabelSomSkalTil}
			%\end{enumerate}																
			\\\hline
	\end{longtable} 
\end{center}