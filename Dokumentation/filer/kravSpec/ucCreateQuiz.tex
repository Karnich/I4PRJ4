% Tabel header

\uchead
	{1}
	{Create quiz}
	{Bruger vil oprette en Quiz.}
	{Bruger trykker på ''Create Quiz'' på brugergrænsefladen.}
	{Bruger, Databaseserver og Webserver.}
	{}
	{1 pr. session.}
	{Bruger har oprettet en Quiz.}

\item Brugergrænsefladen viser de forskellige konfigurationsmuligheder, som Bruger har til rådighed ved ''Create Quiz''.

\item \label{ucCreateQuizName} Bruger indtaster navn på Quiz, og dette tilføjes automatisk som et søgbart tag til Quizzen.

\item Bruger indtaster søgbare tags på Quiz.

\item \label{ucCreateQuizQst} Bruger indtaster spørgsmål.

\item Bruger indtaster svarmuligheder og angiver det korrekte svar.

\item Bruger kan tilføje flere spørgsmål ved at trykke ''Create new question'', hvormed der hoppes til punkt \ref{ucCreateQuizQst}.

\item \label{ucCreateQuizDone} Bruger trykker på ''Create'' knappen.
\textbf{Undtagelse \ref{ucCreateQuizDone}.a} \textbf{Undtagelse \ref{ucCreateQuizDone}.b}

\item \label{ucCreateQuizDB} Den pågælende quiz bliver tilføjet til databasen og UC afsluttes.

\ucdescriptionend

\ucextension
	{Bruger vælger i løbet af use casens forløb at trykke ''Discard Quiz''.}
	{Quizzen oprettes ikke, og use casen afsluttes.}
	{\ref{ucCreateQuizDone}.a}
	
\ucextension
	{De indtastede oplysninger om Quizzen er ikke i overensstemmelse med de ikke-funktionelle krav (se afsnit \ref{sec:nonFunctional}, side \pageref{sec:nonFunctional}).}
	{Gå til UC1.\ref{ucCreateQuizName}}
	{\ref{ucCreateQuizDone}.b}
		
\ucfoot{CreateQuiz}