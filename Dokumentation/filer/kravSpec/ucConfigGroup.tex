\uchead
	{10} % UC-nummer
	{Config Group} % UC-navn
	{Bruger kan slette, ændre tilstand på quizzer i gruppen og slette, tilføje tags i gruppen} % Mål
	{Bruger har trykket på ''Groups''} % Initialisering
	{Bruger, Databaseserver og Webserver} % Aktærer og stakeholders
	{Bruger er logget ind og ejer en gruppe} % Preconditions
	{} % Referencer
	{1 pr. session} % Antal samtidige hændelser
	{Bruger har ændret eller slettet de ønskede quizzer og tags i en given gruppe} % Efterfølgende tilstand

% Tilføj items her. De placeres i hovedforløbet.
% Referencer til punkter laves med \item\label{ucNAVNSomeItem}

\item Bruger trykker på redigeringslogoet for den gruppe der ønskes konfigureret.

\item\label{ucConfigGroupEditGroup} Bruger for vist gruppens navn, quizzer og tags, samt en liste over gruppens medlemmer. \textbf{Undtagelse \ref{ucConfigGroupEditGroup}}

\item Bruger kan vælge at ændre gruppenavn.

\item Bruger kan ændre tags i gruppen.

\item Bruger kan vælge at slette quizzer i gruppen.

\item Bruger kan vælge at aktivere eller at deaktivere en given quiz.

\item Bruger gemmer sine ønskede ændringer.

\item Ændringerne gemmes på databasen.


\ucdescriptionend % Slut på Hovedforløb

\ucextension
	{Bruger navigerer væk fra konfigurationssiden.}
	{UC afsluttes.}
	{\ref{ucConfigGroupEditGroup}.a}

\ucfoot{ucConfigGroup} % Indsæt navn på UC til referencer