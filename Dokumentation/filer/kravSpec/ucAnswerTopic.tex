\begin{center} \centering \label{ucAnswerTopic}
	\begin{longtable}{|p{4.6cm}|p{9.4cm}|}  %% Longtable = forsætter på næste side
		\hline
		\multicolumn{2}{|l|}{\textbf{UC2: Answer Topic}} \\\hline
		\endfirsthead
		
		\multicolumn{2}{l}{...fortsat fra forrige side} \\ \hline %% Til LONGTABLE
		\multicolumn{2}{|l|}{\textbf{UC2: Answer Topic}} \\\hline
		\endhead	
		
		\textbf{Mål}						&Bruger vil besvare en quiz
		\\\hline
		\textbf{Initialisering}			&Bruger har søgt efter quiz
		\\\hline
		\textbf{Aktører og Stakeholders}	&Bruger, Database server og Web server
		\\\hline 
		\textbf{Referencer}				&
		\\\hline
		\textbf{AASH}					&1 pr. session
		\\\hline
		\textbf{Efterfølgende tilstand}	&Bruger har besvaret quiz.
		\\\hline
		\textbf{Hovedforløb}					
			&\begin{enumerate}
			\item Bruger vælger topic
			\item Bruger præsenteres for spørgsmål.
			\item Bruger vælger en af svarmulighederne. 
			\item System viser rigtigt svar.
			\item Næste spørgsmål præsenteres for bruger 
			
			[punkt 2 - 5 gentages indtil der ikke er flere spørgsmål i topic.]
			\item Bruger præsenteres for statistik der viser hvor mange spørgsmål der er 	besvaret korrekt.
			\item Bruger vælger at afslutte hvorved topic forlades (Bruger vises det oprindelige "billede" fra punkt 0)


			\end{enumerate}\\\hline
		\textbf{Undtagelser}
			&			
			% Undtagelser laves som små enumeraters. Brug syntaksen her under, så punkter automatisk opdateres
			%\begin{enumerate}[label=\ref{ucPunktLabel}.a]
			%	\item Gå til UC2.\ref{ucPunktLabelSomSkalTil}
			%\end{enumerate}																
			\\\hline
	\end{longtable} 
\end{center}