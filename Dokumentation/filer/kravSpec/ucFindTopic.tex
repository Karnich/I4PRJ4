\begin{center} \centering \label{ucFindTopic}
	\begin{longtable}{|p{4.6cm}|p{9.4cm}|}  %% Longtable = forsætter på næste side
	\hline
		\multicolumn{2}{|l|}{\textbf{UC3: Find Topic}} \\\hline
		\endfirsthead
		
		\multicolumn{2}{l}{...fortsat fra forrige side} \\ \hline %% Til LONGTABLE
		\multicolumn{2}{|l|}{\textbf{UC3: Find Topic}} \\\hline
		\endhead	
		
		\textbf{Mål}					& Bruger har fundet ønsket topic
		\\\hline
		\textbf{Initialisering}			& (evt. login i senere iteration)
		\\\hline
		\textbf{Aktører og Stakeholders}	&Bruger
		\\\hline 
		\textbf{Referencer}				& (evt. login i senere iteration)
		\\\hline
		\textbf{AASH}					& 1 pr. session
		\\\hline
		\textbf{Efterfølgende tilstand}	& ?
		\\\hline
		\textbf{Hovedforløb}					
			&\begin{enumerate}
				\item Bruger aktiverer søgefeltet på brugergrænsefladen.
				\item \label{ucFindTopicSearch} Bruger indtaster navnet på de tags %(eller evt. navnet på topic??) 
				som det ønskede topic har tilknyttet, og trykker på SØG knappen på brugergrænsefladen eller Enter på brugerens tastatur.
				\item \label{ucFindTopicNoResults} Brugeren vises en række topics, som har tilknyttet de tags brugeren indtastede.
				\textbf{Undtagelse \ref{ucFindTopicNoResults}.a}
				%
				\item Bruger markerer det ønskede topic.
				% \textbf{[Undtagelse \ref{ucPunktLabel}.a]} \newline
				% 	ALERT!!
				
			\end{enumerate}\\\hline
		\textbf{Undtagelser}
			&\begin{enumerate} [label=\ref{ucFindTopicNoResults}.a]
				\item Søgningen fandt ingen topics med de angivne tags %(eller navne??)
				\item Gå til UC3.\ref{ucFindTopicSearch}
			\end{enumerate}
						
			% Undtagelser laves som små enumeraters. Brug syntaksen her under, så punkter automatisk opdateres
			%\begin{enumerate}[label=\ref{ucPunktLabel}.a]
			%	\item Gå til UC3.\ref{ucPunktLabelSomSkalTil}
			%\end{enumerate}																
			\\\hline
	\end{longtable} 
\end{center}