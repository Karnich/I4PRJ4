% Tabel header

\uchead
	{9}
	{Create Group}
	{Bruger vil oprette en gruppe.}
	{Bruger trykker på ''Groups'' på brugergrænsefladens menu.}
	{Bruger, Databaseserver og Webserver.}
	{UC6: Login}
	{1 pr. session.}
	{Bruger har oprettet en gruppe.}
\item Brugeren trykker på ''Create new group''.

\item Webserveren præsenterer de forskellige konfigurationsmuligheder, som Bruger har til rådighed ved ''Create Group'':

\item \label{ucCreateGroupName} Bruger indtaster det ønskede navn på gruppen. \textbf{Undtagelse \ref{ucCreateGroupName}.a}

\item Bruger indtaster søgbare tags separeret med mellemrum.
\item Brugeren kan tilføje sine ejne quizzer til gruppen ved at trykke på checkboksen for de ønskede quizzer.

\item \label{ucCreateGroupCreate} Bruger trykker på ''Create'' knappen, hvorefter gruppen gemmes i databasen.
\textbf{Untagelse \ref{ucCreateGroupCreate}.a}

\item Webserveren præsenterer den nye gruppe.

\ucdescriptionend

\ucextension
	{Der findes allerede en gruppe med det indtastede navn}
	{Der vises en valideringsfejl. Gå til UC9 punkt \ref{ucCreateGroupName}}
	{\ref{ucCreateGroupName}.a}
		
\ucextension
	{Bruger trykker på ''Back'' linket.}
	{Use casen afsluttes. Gruppen gemmes ikke og brugeren præsenteres for samme side, som når der trykkes på ''Groups'' på brugergrænsefladens menu.}
	{\ref{ucCreateGroupCreate}.a}
\ucfoot{CreateGroup}