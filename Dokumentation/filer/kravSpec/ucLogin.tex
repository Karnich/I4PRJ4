% Tabel header

\uchead
	{6}
	{Login}
	{Bruger ønsker at logge ind.}
	{Bruger trykker på ''Login'' på brugergrænsefladen.}
	{Bruger, Databaseserver og Webserver.}
	{}
	{1 pr. session.}
	{Bruger er logget ind.}	
  
		
\item Der vises en login boks med mulighed for at skrive email og password, oprette en ny bruger, samt muligheden for login via facebook. \textbf{Undtagelse \ref{ucLoginSelectMethod}}

\begin{enumerate}

\item \textbf{Login med account}

	\begin{enumerate}
		\item Bruger indtaster email og password til sin oprettede bruger og trykker på knappen ''Sign in'' som ikke indeholder et facebook logo.
		\item \label{ucLoginError}Brugerens data valideres i forhold til data på databaseserveren.\textbf{Undtagelse \ref{ucLoginError}}
		\item Brugerens brugernavn vises i stedet for ''login'' knappen i menuen.
		\item Brugeren er logget ind og use casen afsluttes.
	\end{enumerate}
	
\item \textbf{Login med facebook}
	
	\begin{enumerate}
		\item Brugeren trykker på knappen ''Sign in'' med et facebook logo.
		\item\label{ucLoginWithFacebookError} Brugerens brugernavn vises i stedet for ''login'' knappen i menuen. \textbf{Undtagelse \ref{ucLoginWithFacebookError}}
		\item Brugeren er logget ind og use casen afsluttes.
				
	\end{enumerate}
\end{enumerate}

\ucdescriptionend % Slut på Hovedforløb
	\ucextension
	{Validering af login data fejlede, og der vises en fejlmeddelelse til brugeren }
	{Use casen afsluttes.}
	{\ref{ucLoginError}}

\ucextension
	{Bruger er ikke logget ind på facebook. Brugeren indtaster sine facebook login oplysninger i et facebook-popup vindue.}
	{Gå til UC6 punkt \ref{ucLoginWithFacebookError}.}
	{\ref{ucLoginWithFacebookError}}


\ucfoot{ucConfigQuizzes} % Indsæt navn på UC til referencer