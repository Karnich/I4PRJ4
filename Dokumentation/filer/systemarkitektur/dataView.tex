\section{Data View}

Til systemet bruges persistering af data vha en databse på en databseserver. De oprettede entiteter/tabeller ses på figur \ref{fig:DatabaseDiagram}. Diagrammet benytter relationship-notation i overensstemmelse med Microsoft SQL Management Studio 2014. % Kilde - https://msdn.microsoft.com/en-us/ms188251.aspx
Designet er opbygget på baggrund af domænemodellen fra afsnit \ref{sec:Domainmodel}. Entiteterne er identificeret ud fra domæneklasserne fra domænemodellen, og der er oprettet weak-entities til entiteter med en mange-til-mange relation. Disse har ingen speciel notation på diagrammet, men består af entiteterne ''QuizGroup'', ''QuizTagRelation'', ''UserGroup'', ''TagGroup'' og ''AspNetUserRoles''. Disse weak entities bruges som en slags opslags-entitet, som holder primary key for de to entities den binder.
I diagrammet ses en række entiteter, som alle starter med ''AspNet'', disse entiteter er skabt i forbindelse med oprettelsen af brugere og login, som benytter .NET Identity 2.0 Framework. 
% Kilde - http://www.asp.net/identity

\figur{0.9}{SystemArkitektur/Dataview_Database_diagram}{Diagram over databasens design, som er brugt til systemet}{fig:DatabaseDiagram}

Et databasediagram indeholdende entiteter med over typer, keys og attributter ses i bilaget.
% FIX REFERENCE TIL BILAG! INDSÆT TABEL I BILAG

