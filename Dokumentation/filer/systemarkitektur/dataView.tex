\section{Data View}

Til systemet er der brug for at persistere data om brugere og indhold. For at give et overblik er opdelingen af tabeller i databasen indtegnet i et ER-diagram, som ses på figur \ref{fig:ERdiagram}.


Kardinaliteten er anført i overensstemmelse med de ikke-funktionelle krav (se afsnit \ref{sec:nonFunctional} side \pageref{sec:nonFunctional}).
Der ses på diagrammet at der er en mange-til-mange relation mellem  entiteterne Quiz og Tags samt AspNetUser og AspNetRoles, dette medfører at det er nødvendigt med en weak-entity, som håndterer relationen.

\figur{0.5}{SystemArkitektur/ER_diagram}{ER diagram med Chen notation for databasen}{fig:ERdiagram}

En samlet oversigt over typer, keys og attributter ses i figur \ref{fig:DataView}.

\figur{0.7}{SystemArkitektur/DataStructureTable_QuizCreator}{Tabel over datastrukturen i databasen}{fig:DataView}

