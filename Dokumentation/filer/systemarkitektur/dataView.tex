\section{Data View}

Til systemet bruges persistering af data vha. en database på en databaseserver. De oprettede entiteter/tabeller ses på figur \ref{fig:DatabaseDiagram}. Diagrammet benytter relationship-notation i overensstemmelse med Microsoft SQL Management Studio 2014 \citep{netidentityWeb}.

Designet er opbygget på baggrund af domænemodellen fra afsnit \ref{sec:Domainmodel}, side \pageref{sec:Domainmodel}. Entiteterne er identificeret ud fra domæneklasserne fra domænemodellen, og der er oprettet weak entities til entiteter med en mange-til-mange relation. Disse har ingen speciel notation på diagrammet, men består af entiteterne:

\begin{itemize}
	\item\verb+QuizGroup+
	\item\verb+QuizTagRelation+
	\item\verb+UserGroup+
	\item\verb+TagGroup+
	\item\verb+AspNetUserRoles+
\end{itemize}

Disse weak entities bruges som en slags opslags-entitet, som er nødvendig ved mange til mange relationer. De holder primary keys for de to entiteter den binder sammen.

I diagrammet på figur \ref{fig:DatabaseDiagram} ses en række entiteter, som alle starter med ''\verb+AspNet+'', disse entiteter er skabt i forbindelse med oprettelsen af brugere og login, som benytter .NET Identity 2.0 Framework \citep{netidentityWeb}.

\figur{1}{SystemArkitektur/Dataview_Database_diagram}{Diagram over databasens design, som er brugt til systemet}{fig:DatabaseDiagram}

Entiteternes attributter og typer kan ses i Doxygen-dokumentation. Se afsnit \ref{chap:bilagsCD} på side \pageref{chap:bilagsCD}.