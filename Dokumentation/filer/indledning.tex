\chapter{Indledning}
Tanken med projektet er opbygge en applikation, som vil gøre det let for mennesker imellem at teste hinandens viden. Tests bruges alle steder, både i konkurrencer men også i faglige sammenhænge. 

Meningen med applikationen er at man vil kunne bruge den uanset om man er alene og søger at teste viden indenfor et emne, eller om man er en gruppe af mennesker som søger underholdning til en aften. 
Første prioritet med applikationen er, at den er let og intuitiv at navigere og benytte om enten man er en 4. klasses elev eller IT studerende på en videregående uddannelse. 
Brugerne af applikationen skal være i stand til at kunne logge sig ind og oprette et ''rum'' (hvorved de bliver administrator af rummet), som andre kan tilmelde sig. Her vil kunne oprette sin egen quiz, og herefter tilføje en række spørgsmål som brugeren selv opretter. 

Hvis brugeren ikke selv vil finde på spørgsmål skal han kunne søge i en database, som indeholder alle tidligere oprettede quizzer. Dette gøres ved at hver quiz får såkaldte “tags” - fx. kan man oprette en quiz om division, og give det tagget “division” og “matematik”. 

Administratoren skal kunne se statistikker over sit rums brugere, heriblandt hvor mange spørgsmål de har svaret rigtigt og forkert. En brugers statistik skal også være tilgængelig for brugeren selv - Der skal være mulighed for at udbygge denne statistik grafisk, f.eks i form af bjergbestigning eller pokaler, der stiger som brugeren svarer rigtigt på flere spørgsmål i et givet rum.

Hvert spørgsmål skal kunne have en rating, sådan at ved søgning af spørgsmål kan der sorteres efter hvor gode spørgsmålene er.

På figur \ref{fig:overblik} ses et overblik over rum, brugere, quizzer og spørgsmål.

\figur{0.5}{Indledning/opbygningOverblik.pdf}{Skitsering af produktets opbygning}{fig:overblik}