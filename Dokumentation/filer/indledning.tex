\chapter{Indledning}
Tanken med projektet er opbygge en web applikation, som vil gøre det let for mennesker imellem at teste hinandens viden. Tests bruges alle steder, både i konkurrencer men også i faglige sammenhænge. 

Meningen med applikationen er at man vil kunne bruge den uanset om man er alene og søger at teste viden indenfor et emne, eller om man er en gruppe af mennesker som søger underholdning til en aften. 
Første prioritet med applikationen er, at den er let og intuitiv at navigere og benytte om enten man er en 4. klasses elev eller IT studerende på en videregående uddannelse. 
Brugerne af applikationen skal være i stand til at kunne logge sig ind og oprette en ''gruppe'' (hvorved de bliver administrator af gruppen), som andre kan tilmelde sig. Her vil kunne oprette sin egen quiz, og herefter tilføje en række spørgsmål som brugeren selv opretter. 

Hvis brugeren ikke selv vil finde quizzer skal han kunne søge i en database, som indeholder alle tidligere oprettede quizzer. Dette gøres ved at hver quiz får såkaldte “tags” - fx. kan man oprette en quiz om division, og give det tagget “division” og “matematik”. En quiz kan således ligge frit tilgængeligt, uden at være tilknyttet en gruppe. Bruger man web applikationen uden er være logget ind kan man ikke kunne tilknyttes grupper, men har stadig mulighed for at besvare quizzer, og se sit resultat.

På figur \ref{fig:overblik} ses et overblik over grupper, brugere, quizzer og spørgsmål.

\figur{0.5}{Indledning/opbygningOverblik.png}{Skitsering af produktets opbygning}{fig:overblik}