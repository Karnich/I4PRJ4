% Ret emne og skriv indhold
\section{Grafisk brugerflade-teknologier}

% Indhold
Følgende teknologier er vurderet i forbindelse med udviklingen af den grafiske brugergrænseflade.

\begin{tabular}{|p{3cm}|p{5.5cm}|p{5.5cm}|}
\hline 
\textbf{Navn} & \textbf{Pros} & \textbf{Cons} \\ 
\hline

Microsoft\newline WFP
	&
	\begin{itemize}
		\item Anvendes i I4GUI
	\end{itemize}
	&
	\begin{itemize}
		\item Kan kun bruges på Windows-platfomen
		\item Kræver .NET-frameworket på klientens computer
	\end{itemize}
	\\
	\hline

Microsoft\newline Silverlight
	&
	\begin{itemize}
		\item Stærkt framework til grafisk præsentation
	\end{itemize}
	&
	\begin{itemize}
		\item Udvikles ikke længere af Microsoft
		\item Kræver Silverlight-\newline frameworket på klientens\newline computer
	\end{itemize}
	\\
	\hline

HTML5 og CSS3
	&
	\begin{itemize}
		\item Anvendes i I4GUI
		\item Den gængse teknologi for web-applikationer
		\item Understøttet af de fleste platforme
	\end{itemize}
	&
	\begin{itemize}
	\item Mindre grafiske evner
	\end{itemize}
	\\
	\hline
\end{tabular} 

Ud fra ovenstående er HTML5 og CSS3 valgt for at få en bredt understøttet teknologi som har rigelige muligheder for programmets behov og anvendes generelt mest, af de belyste teknologier.