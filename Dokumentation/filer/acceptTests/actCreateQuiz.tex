\achead{UC1}{Create Quiz}

% Testpunkter
% \acentry{Test punkt iht. UC beskrivelse}{Test}{Forventet resultat}{Resultat}{Godkendt eller kommentar}

\acentry{1}{\begin{enumerate}
			\item Udfør alle interne tests i \textbf{Test 0: Generel InputTest} Generel InputTest udføres på de følgende tekstbokse:
				\subitem Quizname
				\subitem Tags
				\subitem Question 1
				\subitem Answer 1
				\subitem Answer 2
			\end{enumerate}		   	
		   	}
		   	{ Alle interne tests i Test 0 er gennemført og godkendt }{ - }{ - }


\acentry{2}{ \begin{enumerate}
			\item Udfør accepttesten for UC4: Create User test 1 for at oprette en user. 
			\item Udfør accepttesten UC6: Login test 1 for at logge ind. 
			\item Tryk på "Quizzes" og derefter vælg "Create new quiz". 
			\item Indtast følgende værdier:
			\subitem Quizname: \textbf{TestQuiz}.
			\subitem Tags: \textbf{tag1}, \textbf{tag2}
			\subitem Question 1: \textbf{Spørgsmål 1}
			\subitem Answer 1: \textbf{Rigtigt svar} med check i boxen ''Correct answer'' og angiver points til 10.
			\subitem Answer 2: \textbf{Forkert svar} uden check i boxen ''Correct answer'' og angiver points til 0.
			\item Der trykkes på ''Create Quiz''.
			\end{enumerate}
			}
			{Quiz med 1 spørgsmål indeholdende 2 svarmuligheder med de indtastede navne og værdier er oprettet i databasen på database serveren.}{ - }{ - }

\acentry{3}{ \begin{enumerate}
			\item Udfør accepttesten for UC4: Create User test 1 for at oprette en user. 
			\item Udfør accepttesten UC6: Login test 1 for at logge ind. 
			\item Tryk på "Quizzes" og derefter vælg "Create new quiz". 
			\item Indtast følgende værdier:
			\subitem Quizname: \textbf{TestQuiz2}.
			\subitem Tags: \textbf{tag2}, \textbf{tag3}
			\subitem Question 1: \textbf{Spørgsmål 1}
			\subitem Answer 1: \textbf{Rigtigt svar} med check i boxen ''Correct answer'' og angiver points til 10.
			\subitem Answer 2: \textbf{Forkert svar} uden check i boxen ''Correct answer'' og angiver points til -5.
			\item Der trykkes på ''Add Answer'', og følgende værdier indtastes:
			\subitem Answer 3: \textbf{Semirigtigt svar} med check i boxen ''Correct answer'' og angiver points til 5.
			\item Der trykkes på ''Gennemse...'' og en billedfil med navnet \textbf{billede.jpg} vælges.
			\item Der trykkes på ''Create Quiz''.
			\end{enumerate}
			}
			{Quiz med 1 spørgsmål indeholdende 3 svarmuligheder med de indtastede navne og værdier er oprettet i databasen på database serveren samt stien til billedfilen. Billedfilen er uploadet til webserveren}{ - }{ - }

\acfoot{createQuiz}