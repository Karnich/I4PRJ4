\subsubsection{Create Group}
For at oprette en ny gruppe bruges Create Group viewet. Det findes ved at trykke på ''Groups'' og her efter trykke på knappen ''Create new group'' i øvre højre hjørne på figur \fxnote{BS: Indsæt reference til view my groups viewet}. Det er nødvendigt at være logget ind for at kunne se knappen ''Groups''.

Create Group viewet har en simpel inputform til at oprette en ny gruppe med navn og tags som vist på figur \fxnote{BS: Indsæt reference til Create Group screenshot}. Der ud over er det muligt at tilføje brugerens egne quizzer med det samme fra listen ud for ''Add quizzes''.

Der er implementeret en validering af gruppenavnet, som sikrer at brugeren ikke oprette en gruppe med et navn der allerede er registreret. Til dette anvendes ASP.NET-frameworkets Remote Validation funktion, se afsnit \ref{sec:remoteValidation} på side \pageref{sec:remoteValidation}.

\fxnote{BS: Indsæt Create Group screenshot}