\subsubsection{Config User}

En bruger med en brugerprofil kan til hver en tid konfigurere sin profils informationer. Dog er det ikke muligt for en bruger at ændre sit brugernavn. Grunden til dette er, at de quizzer og grupper, som en bruger opretter, refererer til den pågældende brugers brugernavn. Det betyder altså, at når en bruger opretter en ny brugerprofil, vil det pågældende brugernavn være lås til brugeren. 

Hvis en bruger ønsker at ændre sin adgangskode, skal brugeren oplyse den nuværende aktive adgangskode og derefter indtaste en ny adgangskode. For at ændre en brugers adgangskode anvendes ChangePassword() metoden fra UserManager klassen \citep{usermanager}, som er en del af ASP.NET Identity framework. ChangePassword() metoden tager tre parametre, bruger ID, nuværende adgangskode og den nye adgangskode. Men før der ændres på adgangskoden, skal der først undersøges om brugerens input af den nuværende adgangskode er gyldig. Til det anvendes VerifyHashedPassword() metoden, som også er en del af UserManager klassen. VerifyHashedPassword() hasher brugerens input af den nuværende adgangskode og sammenligner det med den nuværende originale hashede adgangskode. 

Figur \ref{fig:ConfigUserView} viser Config User viewets opbygning. 

\figur{0.8}{Implementering/ViewConfigUser}{GUI af Config User}{fig:ConfigUserView}
