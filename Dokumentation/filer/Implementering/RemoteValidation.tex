Mange forme på siden er udviklet til at kunne udvides dynamisk ved brug af AJAX og jQuery. I de controllere hvor funktionaliteten er nødvendig, som for eksempel CreateQuiz og ConfigQuiz, er der metoder som returnere den nødvendige HTML opbygning. Denne hentes med AJAX og indsættes på siden uden at hele siden behøver at loades igen.

For at udnytte \verb+ASP.NET+s evne til at arbejde på modeller redigeres navenene på alle inputfelterne så de svare til de navne der ville være hvis objekterne var oprettet i \verb+C#+.
For eksempel, hvis en quiz har ét spørgsmål og der tilføjes et nyt, får det nye input-felt, til teksten, navnet \verb+Questions[1].Text+.
På denne måde kan den samlede håndtering af form-inputtet bruge ét argument af typen \verb+Quiz+, i stedet for at skulle tage alle parametre ind én af gangen. Et udklip af koden er vist i listing \ref{lst:QuizzesJavascript}.

På linje 2-5 er AJAX-kaldet. Hvis dette går godt udføres funktionen i linje 5. Her tilføjes HTML-koden fra controlleren en \verb+div+-container. Model-binding navnet konstrueres og tilføjes alle nye \verb+input+-tags. Til sidst returneres false, for at undgå at formens almindelige submit-action udføres.

\lstsetjavascript
\begin{lstlisting}[caption=JavaScript udklip til AJAX og model-binding håndtering ved indsættelse af nye spørgsmål, label=lst:QuizzesJavascript]
$(document).on("click", "#addQuizQuestion", function() {
    $.ajax({
        url: this.href,
        cache: false,
        success: function(html) {
            // Append new question form
            $("#ConfigQuizzesQuestionsContainer").append(html);

            /* Removed code for simplicity */
			
            // Create prepend string to fix object binding
            var prePend = "Questions[" + (newCounter - 1) + "].";

            // Get all input elements and prepend input names
            $(".ConfigQuizzesQuestion:last input").each(function() {
                var original = $(this).attr("name");
                $(this).attr("name", prePend + original);
            });
            
            /* Removed code for simplicity */
        }
    });
    return false;
});
\end{lstlisting}