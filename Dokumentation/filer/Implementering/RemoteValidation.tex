\section{Remote Validation}\label{sec:remoteValidation}
Der er implementeret Remote Validation til at finde ud af om gruppe- og brugernavne er taget eller ej. Dette fungerer ved at viewet via model-binding kigger på Groupname- og Username attributerne og ser at der er opsat remote validation til attributerne. Der bliver kaldt en validation metode i RemoteValidationControlleren som så finder ud af om det navn brugeren har indtastet, allerede er taget eller ej. Hvis navnet er ledigt returneres true og intet udskrives i viewet. Men hvis navnet er optaget returneres der false og den givne fejlbesked bliver udskrevet. Eksempel på en ErrorMessage kan ses på figur \ref{fig:remoteValidation}

\figur{1.0}{Implementering/RemoteValidering}{Viser Remote Validation notation på modellen og funktionen den kalder}{fig:remoteValidation}

På figur \ref{fig:remoteValidation} ses der implementeringen af remote validering til group name. [Remote ...] ligger i System.Web.Mvc.RemoteAttribute namespacet og er det som der kalder valideringsmetoden DoesGroupNameExist.