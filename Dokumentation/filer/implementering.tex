\chapter{Implementering}

Her følger detaljerede beskrivelser af implementeringen af software.

\section{Detaljeret beskrivelse af systemet med Doxygen}
Til at dokumentere metoder og klasser, gøres der brug af Doxygen. Ved at tilføje tre skråstreger (///) over klasser og metoder, kan Doxygen generere en html side hvori beskrivelser og relationer er vist.
På \ref{fig:doxygen} ses et eksempel på beskrivelse af en metode.

\figur{1}{Implementering/DoxygenExample.png}{Eksempel på Doxygen beskrivelse i koden}{fig:doxygen}

På Cden følger der en html side under Doxygen, hvor dokumentationen præsenteres. På \ref{fig:doxygenhtml}, ses et screenshot af layoutet.

\figur{0.6}{Implementering/DoxygenHtml.png}{Visning af Doxygen layout}{fig:doxygenhtml}

\section{Beskrivelse af views}\label{sec:views}
Her følger beskrivelser af de views der præsenteres på websiden.
\subsection{Quiz views}
\subsubsection{Create Quiz}
I Create Quiz viewet kan der oprettes en ny quiz. Viewet er delt op således at Question og Answer er to partialviews, som er fælles for Create Quiz og Config Quiz.
\fxnote{Lasse: Indsæt billede af Create Quiz viewet, tilføj grafiske bokse der viser hvad der er partial view (Question, Answer)}

I Question partial viewet bliver der brugt Bootstrap paging, det fungere ved at alle divs bliver listet ud efter hindanden, men Bootstrap sørger for kun at vise det der har css-klassen "active" på sig.

\subsubsection*{Dynamisk generering af inputfelter i forme}
Formene i Create og Config quiz er udviklet til at kunne udvides dynamisk ved brug af AJAX og jQuery. I controllerne CreateQuiz og ConfigQuiz, er der metoder som returnerer den nødvendige HTML opbygning. Denne hentes med AJAX og indsættes på siden uden at hele siden behøver at loades igen.

For at udnytte ASP.NETs evne til at arbejde på modeller redigeres navenene på alle inputfelterne så de svarer til de navne der ville være hvis objekterne var oprettet i C\#.
For eksempel, hvis en quiz har ét spørgsmål og der tilføjes et nyt, får det nye input-felt, til teksten, navnet \verb+Questions[1].Text+.
På denne måde kan den samlede håndtering af form-inputtet bruge ét argument af typen \verb+Quiz+, i stedet for at skulle tage alle parametre ind én af gangen. Et udklip af koden er vist i \lstlistingname \ref{lst:QuizzesJavascript}.

På linje 2-5 er AJAX-kaldet. Hvis dette går godt udføres funktionen i linje 5. Her tilføjes HTML-koden fra controlleren en \verb+div+-container. Model-binding navnet konstrueres og tilføjes alle nye \verb+input+-tags. Til sidst returneres false, for at undgå at formens almindelige submit-action udføres.

\lstsetjavascript
\begin{lstlisting}[caption=JavaScript udklip til AJAX og model-binding håndtering ved indsættelse af nye spørgsmål, label=lst:QuizzesJavascript]
$(document).on("click", "#addQuizQuestion", function() {
    $.ajax({
        url: this.href,
        cache: false,
        success: function(html) {
            // Append new question form
            $("#ConfigQuizzesQuestionsContainer").append(html);

            /* Removed code for simplicity */
			
            // Create prepend string to fix object binding
            var prePend = "Questions[" + (newCounter - 1) + "].";

            // Get all input elements and prepend input names
            $(".ConfigQuizzesQuestion:last input").each(function() {
                var original = $(this).attr("name");
                $(this).attr("name", prePend + original);
            });
            
            /* Removed code for simplicity */
        }
    });
    return false;
});
\end{lstlisting}
\subsubsection*{Sletning af spørgsmål og svar}
Udover dynamisk generering af spørgsmål og svar, kan de også slettes igen. Dette gøres med JQuery, hvor hele klientens html bliver manipuleret, og på den måde kan der vælges hvilke divs der skal være "hidden". Derudover skal der rettes i indekseringen af spørgsmål og svar, så ledes at hvis man sletter spørgsmål 1, skal spørgsmål 2 nu blive til nummer 1 ovs.
\subsubsection{Config Quizzes}

%Beskrivelse af viewet (det er næsten det samme som Create Quiz)

%Beskrivelse af DeleteFlags der sættes på Question, Answer og File
\subsubsection{View My Quizzes}
Viewet åbnes ved at trykke på ''Quizzes'' i menuen som ses på figur \ref{fig:MyQuizzesView}. Det kræver at brugeren er logget ind for at åbne viewet. Hvis det skulle ske, at en bruger der ikke er logget ind forsøger at åbne viewet bliver han returneret til websidens index.
Dette view danner en bro mellem brugeren og de quizzer som brugeren selv har oprettet. Viewet tillader brugeren at slette eller ændre sine quizzer. Ved tryk på blyanten under Edit åbnes Config Quizzes viewet for denne quiz. Quiznavnet agerer som link til Answer Quiz viewet med denne quiz. 

\figur{0.9}{Implementering/ViewMyQuizzes}{GUI af my quizzes viewet}{fig:MyQuizzesView}
\subsubsection{Answer Quiz}
I Answer Quiz, kan der markeres forskellige svar muligheder inden for hvert spørgsmål. Spørgsmålene bliver præsenteret med Bootstrap paging, således at der er en list af knapper hvor det ønskede spørgsmål i rækken kan vælges. Derudover kan der også bladres igennem spørgsmålene med en "Previous" og "Next" knap. Knapperne er tilkoblet en JQuery funktion der sætter en "active" klasse på det hhv. forrige og næste spørgsmåls div.

\fxnote{Lasse: Tilføje screenshot af Answer Quiz viewet}
\subsubsection{Search}

På menubjælken er der et inputfelt hvor bruger kan søge efter quizzer og/eller grupper ved at indtaste ønskede søgeargument. Når bruger trykker på "SØG" knappen bliver bruger præsenteret for Search viewet. Search viewet viser en række quizzer og/eller grupper som matcher brugers søgeargument. Hvis ingen quiz eller gruppe blev fundet vil Search viewet informere bruger om det. 

Søgesystemet søger  efter quizzer og grupper som har tags der matcher søgeargumentet - quiz navn og gruppe navn indgår også som tag. Søgesystemet har en meget bredt søgning. Dvs. at så længe søgeargumentet indgår i en eller flere quizzer og/eller gruppers tags, vil de pågældende quizzer og/eller grupper bliver præsenteret på Search viewet. Fx hvis søgeargumentet er 'd', vil søgesystemet finde samlige quizzer og/eller grupper hvor 'd' indgår.  


Search viewet består i princippet af to views. Et som viser alle de quizzer som matcher søgeargumentet og det andet viser grupperne. Til visning af de to views bruges bl.a. jQuery AJAX teknologien. Ved brug af jQuery AJAX kan der skiftes imellem de to views uden at hele web-applikationen bliver opdateret, fordi client-side sender og modtager data til og fra server-side asynkron, dvs. at der opdateres kun på det ændrede data og ikke hele web-applikationen. 




\subsection{Group views}
\subsubsection{Create Group}
For at oprette en ny gruppe bruges Create Group viewet. Det findes ved at trykke på ''Groups'' og her efter trykke på knappen ''Create new group'' i øvre højre hjørne på figur \fxnote{BS: Indsæt reference til view my groups viewet}. Det er nødvendigt at være logget ind for at kunne se knappen ''Groups''.

Create Group viewet har en simpel inputform til at oprette en ny gruppe med navn og tags som vist på figur \fxnote{BS: Indsæt reference til Create Group screenshot}. Der ud over er det muligt at tilføje brugerens egne quizzer med det samme fra listen ud for ''Add quizzes''.

Der er implementeret en validering af gruppenavnet, som sikrer at brugeren ikke oprette en gruppe med et navn der allerede er registreret. Til dette anvendes ASP.NET-frameworkets Remote Validation funktion, se afsnit \ref{sec:remoteValidation} på side \pageref{sec:remoteValidation}.

\fxnote{BS: Indsæt Create Group screenshot}
\subsubsection{Config Group}
Dette view giver brugeren mulighed for at ændre i informationerne for en gruppe. Billedet er næsten identisk med ''Create Group'' viewet. Det er muligt at ændre de tilknyttede tags og det er muligt at fjerne quizzer og tilføjer egne quizzer ud fra to lister. Det er ikke muligt at ændre gruppens navn, da dette er besluttet at være en bindende beslutning når gruppen oprettes. Se figur \fxnote{BS: Indsæt reference til Config Group figur}.

\fxnote{BS: Indsæt screenshots af Config Group}
\subsubsection{View My Groups}
Viewet åbnes ved at trykke på ''Groups'' i menuen som ses på figur \ref{fig:MyGroupsView}. Det kræver at brugeren er logget ind for at åbne viewet. Hvis det skulle ske, at en bruger, der ikke er logget ind forsøger at åbne viewet bliver denne returneret til websidens index.
Viewet tillader brugeren at benytte de grupper, som brugeren selv har oprettet samt de grupper, som brugeren har tilmeldt sig. Viewet tillader brugeren at ændre sine grupper ved at trykke på blyanten under Edit, hvor Config Group viewet for denne gruppe åbnes. Dette er kun muligt for de grupper som brugeren selv har oprettet. Gruppenavnet agerer som link til View Group viewet med denne gruppe. 

\figur{0.8}{Implementering/ViewMyGroups}{GUI til my Groups viewet}{fig:MyGroupsView}
\subsubsection{View Group}
Dette view åbnes når man trykker på en gruppe i View My Groups viewet eller har fundet en gruppe med søgefunktionen.

Viewet er som vist på figur \ref{fig:ViewGroup}. Den viser to kolonner med de tilknyttede quizzer og medlemmerne i gruppen. Ved at trykke på quiznavnene åbnes Answer Quiz for den pågældende quiz. Det er også muligt at farvorisere gruppen ved at klikke på stjernen ved siden af navnet på gruppen. 

\figur{0.8}{Implementering/ViewGroup}{GUI til group viewet}{fig:ViewGroup}
\subsection{User views}
Til at håndtere brugerprofiler, har vi anvendt ASP.NET Identity framework 2.0 \citep{msdnIdentityFrameworkWeb}. ASP.NET Identity er et bibliotek, som har mange brugbare klasser, når det gælder håndtering af brugerprofiler. Klasser som gør det muligt at oprette en ny bruger med en bestemt adgangskode, slette en bruger, opdatere en bruger, ændre adgangskoden til en bruger, sende en E-mail og/eller SMS til en bruger og meget mere. Et argument for at anvende ASP.NET Identity framework, er at det selv sørger for at hashe brugerens adgangskode, så denne ikke fremgår direkte i databasen.

\subsubsection{Create User}

Create User viewet kan tilgås ved at klikke på ''Login'' og derefter vælge ''Register''. Inde på Create User viewet kan en bruger oprette en ny brugerprofil. Viewet indeholder en række inputfelter, hvor brugeren skal indtaste og oplyse personlige informationer. Der er krav om at brugeren skal vælge et brugernavn og adgangskode samt oplyse en gyldig email-adresse. Derudover kan brugeren også oplyse fornavn og efternavn, men dette er ikke et krav. 
Til oprettelse af en bruger med en ny bestemt adgangskode anvendes Create() metoden fra UserManager klassen. Create metoden tager to parametre: Den nye brugerprofil, samt adgangskoden til den pågældende brugerprofil. 

Figur \ref{fig:CreateUserView} viser Create User viewet. 

\figur{0.8}{GUI/CreateUserView}{Create User view}{fig:CreateUserView}

\subsubsection{Config User}

En bruger med en brugerprofil kan til hver en tid konfigurere sin profils informationer. Dog er det ikke muligt for en bruger at ændre sit brugernavn. Grunden til dette er, at de quizzer og grupper, som en bruger opretter, refererer til den pågældende brugers brugernavn. Det betyder altså, at når en bruger opretter en ny brugerprofil, vil det pågældende brugernavn være lås til brugeren. 

Hvis en bruger ønsker at ændre sin adgangskode, skal brugeren oplyse den nuværende aktive adgangskode og derefter indtaste en ny adgangskode. For at ændre en brugers adgangskode anvendes ChangePassword() metoden fra UserManager klassen \citep{usermanager}, som er en del af ASP.NET Identity framework. ChangePassword() metoden tager tre parametre, bruger ID, nuværende adgangskode og den nye adgangskode. Men før der ændres på adgangskoden, skal der først undersøges om brugerens input af den nuværende adgangskode er gyldig. Til det anvendes VerifyHashedPassword() metoden, som også er en del af UserManager klassen. VerifyHashedPassword() hasher brugerens input af den nuværende adgangskode og sammenligner det med den nuværende originale hashede adgangskode. 

Figur \ref{fig:ConfigUserView} viser Config User viewets opbygning. 

\figur{0.8}{Implementering/ViewConfigUser}{GUI af Config User}{fig:ConfigUserView}

\subsection{Favorite views}
\subsubsection{View Favorites}
Favorites viewet er brugerens hurtig-adgang til quizzer og grupper. Brugeren kan tilføje favorit grupper og quizzer fra andre views, og derefter kunne tilgå disse i dette view. Dette kan være favorabelt frem for at søge efter quizzer og grupper hver gang de skal tilgås. \newline
Viewet deles op i to partial views: Èt for favoritgrupper og èt for favoritquizzer, hvor favoritgrupper er det partialview, som vises når viewet åbnes via menuknappen. Viewet med sine to partialviews ses i figur \fxnote{Lars: Indsæt to view-billeder af favorites - 1 for favoritgrupper, 1 for favoritquizzer}.
Quiz- og gruppenavnene agerer som links til hhv. Answer Quiz og View Group views. Favoritquizzer og -grupper kan fjernes ved at trykke på knappen i ''remove'' kolonnen.

\section{Remote Validation}
Der er implementeret Remote Validation til at finde ud af om gruppe- og usernavne er taget eller ej. Dette fungere ved at viewet via model-binding kigger på Groupname og Username attributerne og ser at der er opsat remote validation til attributerne. Der bliver kaldt en validation metode i RemoteValidationControlleren som så finder ud af om det navn brugeren har indtastet, allerede er taget eller ej. Hvis navnet er ledigt returneres true og intet udskrives i viewet. Men hvis navnet er optaget returneres der false og den givne fejlbesked bliver udskrevet. Eksempel på en ErrorMessage kan ses på figur \ref{fig:remoteValidation}

\figur{1.0}{Implementering/RemoteValidering}{Viser Remote Validation notation på modellen og funktionen den kalder}{fig:remoteValidation}

På figur \ref{fig:remoteValidation} ses der implementeringen af remote validering til group name. [Remote ...] ligger i System.Web.Mvc.RemoteAttribute namespacet og er det som der kalder validerings metoden DoesGroupNameExist.
