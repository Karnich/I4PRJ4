\chapter{Implementering}

Her følger detaljerede beskrivelser af implementeringen af software.

\section{Detaljeret beskrivelse af systemet med Doxygen}
Til at dokumentere metoder og klasser, gøres der brug af Doxygen. Ved at tilføje tre skråstreger (///) over klasser og metoder, kan Doxygen generere en html side hvori beskrivelser og relationer er vist.
På \ref{lst:doxygen} ses et eksempel på beskrivelse af en metode.

\lstsetcsharp
\begin{lstlisting}[caption=Eksempel på Doxygen notation, label=lst:doxygen]
/// <summary>
/// Get quiz by Id from QuizModel, aggregate tags to string and return ConfigQuiz view
/// </summary>
/// <param name="id">Quiz ID</param>
/// <returns>Editor view for the quiz. Redirects to Index() if Id is null or non-positive</returns>
public ActionResult ConfigQuiz(int? id)
{
	if (!User.Identity.IsAuthenticated)
		return RedirectToAction("Index", "Home");
	...
}
\end{lstlisting}

På Cden følger der en html side under Doxygen, hvor dokumentationen præsenteres. På \ref{fig:doxygenhtml}, ses et screenshot af layoutet.

\figur{0.6}{Implementering/DoxygenHtml.png}{Visning af Doxygen layout}{fig:doxygenhtml}

\section{Beskrivelse af views}\label{sec:views}
Her følger beskrivelser af de views der præsenteres på websiden.
\subsection{Quiz views}
\subsubsection{Create Quiz}

\subsubsection{Config Quizzes}

\subsubsection{View My Quizzes}
Viewet åbnes ved at trykke på ''Quizzes'' i menuen som ses på figur \fxnote{Lars: Indsæt billede af MyQuizzes viewet}. Det kræver at brugeren er logget ind for at åbne viewet. Hvis det skulle ske, at en bruger der ikke er logget ind forsøger at åbne viewet bliver han returneret til websidens index.
Dette view danner en bro mellem brugeren og de quizzer som brugeren selv har oprettet. Viewet tillader brugeren at slette eller ændre sine quizzer. Ved tryk på blyanten under Edit åbnes Config Quizzes viewet for denne quiz. Quiznavnet agerer som link til Answer Quiz viewet med denne quiz. 
\subsubsection{Answer Quiz}
I Answer Quiz, kan der markeres forskellige svar muligheder inden for hvert spørgsmål. Spørgsmålene bliver præsenteret med Bootstrap paging, således at der er en list af knapper hvor det ønskede spørgsmål i rækken kan vælges. 

\fxnote{Lasse: Tilføje screenshot af Answer Quiz viewet}
\subsubsection{Search}

På menubjælken er der et inputfelt hvor bruger kan søge efter quizzer og/eller grupper ved at indtaste ønsket søgeargument. Når bruger trykker på "SØG" knappen bliver bruger præsenteret for Search viewet. Search viewet viser en række quizzer og/eller grupper som matcher brugers søgeargument. Hvis ingen quiz eller gruppe blev fundet vil Search viewet informere bruger om det. 

Søgesystemet søger  efter quizzer og grupper som har tags der matcher søgeargumentet - quiznavn og gruppenavn indgår også som tag. Søgesystemet har en meget bred søgning. Dvs. at så længe søgeargumentet indgår i en eller flere quizzer og/eller gruppers tags, vil de pågældende quizzer og/eller grupper bliver præsenteret på Search viewet. Fx hvis søgeargumentet er 'd', vil søgesystemet finde samlige quizzer og/eller grupper hvor 'd' indgår.  


Search viewet består i princippet af to views. Et som viser alle de quizzer som matcher søgeargumentet og det andet viser grupperne. Til visning af de to views bruges bl.a. jQuery AJAX teknologien. Ved brug af jQuery AJAX kan der skiftes imellem de to views uden at hele web applikationen bliver opdateret, fordi client side sender og modtager data til og fra server side asynkron, dvs. at der opdateres kun på det ændrede data og ikke hele web applikationen. 

Figur \ref{fig:QuizView} viser viewet for Quizzes, og figur \ref{fig:GroupView} viser viewet for Groups. 

\figur{0.8}{GUI/QuizSearchView}{Search view - viser Quizzer}{fig:QuizView}

\figur{0.8}{GUI/GroupSearchView}{Search view - viser Groups}{fig:GroupView}

\subsection{Group views}
\subsubsection{Create Group}
For at oprette en ny gruppe bruges Create Group viewet. Det findes ved at trykke på ''Groups'' og herefter trykke på knappen ''Create new group'' i øvre højre hjørne på figur \ref{fig:MyGroupsView}. Det er nødvendigt at være logget ind for at kunne se knappen ''Groups''.

Create Group viewet har en simpel inputform til at oprette en ny gruppe med navn og tags som vist på figur \ref{fig:ViewCreateGroup}. Derudover er det muligt at tilføje brugerens egne quizzer med det samme fra listen ud for ''Add quizzes''.

Der er implementeret en validering af gruppenavnet, som sikrer at brugeren ikke opretter en gruppe med et navn der allerede er registreret. Til dette anvendes ASP.NET-frameworkets Remote Validation funktion, se afsnit \ref{sec:remoteValidation} på side \pageref{sec:remoteValidation}.

\figur{0.8}{Implementering/ViewCreateGroup}{GUI til create user viewet}{fig:ViewCreateGroup}
\subsubsection{Config Group}

\subsubsection{View My Groups}
Viewet åbnes ved at trykke på ''Groups'' i menuen som ses på figur \ref{fig:MyGroupsView}. Det kræver at brugeren er logget ind for at åbne viewet. Hvis det skulle ske, at en bruger, der ikke er logget ind forsøger at åbne viewet bliver denne returneret til websidens index.
Viewet tillader brugeren at benytte de grupper, som brugeren selv har oprettet samt de grupper, som brugeren har tilmeldt sig. Viewet tillader brugeren at ændre sine grupper ved at trykke på blyanten under Edit, hvor Config Group viewet for denne gruppe åbnes. Dette er kun muligt for de grupper som brugeren selv har oprettet. Gruppenavnet agerer som link til View Group viewet med denne gruppe. 

\figur{0.8}{Implementering/ViewMyGroups}{GUI til my Groups viewet}{fig:MyGroupsView}
\subsubsection{View Group}
\subsection{User views}
Til at håndtere brugerprofiler, har vi anvendt ASP.NET Identity-frameworket \citep{msdnIdentityFrameworkWeb}. ASP.NET Identity er et bibliotek, som har mange brugbare klasser, når det gælder håndtering af brugerprofiler. Klasser som gør det muligt at oprette en ny bruger med en bestemt adgangskode, slette en bruger, opdatere en bruger, ændre adgangskoden til en bruger, sende en E-mail og/eller SMS til en bruger og meget mere. Et argument for at anvende ASP.NET Identity framework, er at det selv sørger for at hashe brugerens adgangskode, så denne ikke fremgår direkte i databasen.

\subsubsection{Create User}

Create User viewet kan tilgås ved at klikke på "Login" og derefter vælge "Register". Inde på Create User kan en bruger oprette en ny brugerprofil. På dette view er der inputfelter, hvor bruger skal indtaste og oplyse personlig informationer. Der er krav om at bruger skal vælge et "Username" og "Password" samt oplyse et gyldigt "E-mail". Derudover kan bruger også oplyse fornavn, "Firstname" og efternavn, "Lastname", men det er ikke et krav. 
Til oprretelse af en bruger med en ny bestemt adgangskode anvendes Create() metoden fra UserManager klassen, som er en del af ASP.NET Identity framework. Create metoden tager to parameter, Den nye brugerprofil og den informationer samt adgangskoden til den pågældende brugerprofil. 

Figur \ref{fig:CreateUserView} viser Create User viewet. 

\figur{0.8}{GUI/CreateUserView}{Create User view}{fig:CreateUserView}

\subsubsection{Config User}

En bruger med en brugerprofil kan til hver en tid rekonfigurere sin profils informationer. Dog er det ikke muligt for en bruger at ændre sit brugernavn, "Username". Grunden til dette, er at de quizzer og grupper som en bruger opretter refererer til den pågældende brugers "Username". Dvs. når en bruger opretter en ny brugerprofil, vil det pågældende "Username" være lås til brugeren. 

Hvis en bruger ønsker at ændre sin adgangskode, "Password", skal brugeren oplyse den nuværende aktive adgangskode og derefter indtaste en ny adgangskode. For at ændre en brugers adgangskode anvendes ChangePassword() metoden fra UserManager klassen, som er en del af ASP.NET Identity framework. ChangePassword() metoden tager tre parametre, bruger ID, nuværende adgangskode og den nye adgangskode. Men før der ændres på adgangskoden, skal der først tjekkes om brugerens input af den nuværende adgangskode er gyldig. Til det anvendes VerifyHashedPassword() metoden, som også er en del af UserManager klassen. VerifyHashedPassword() hasher brugerens input af den nuværende adgangskode og sammenligner det med den nuværende originale hashede adgangskode. 

Figur \ref{fig:ConfigUserView} viser Config User viewet. 

\figur{0.8}{GUI/ConfigUserView}{Config User view}{fig:ConfigUserView}

\subsection{Favorite views}
\subsubsection{View Favorites}
Favorites viewet er brugerens hurtig-adgang til quizzer og grupper. Brugeren kan tilføje favoritgrupper og -quizzer fra andre views, og derefter kunne tilgå disse i dette view. Dette kan være favorabelt frem for at søge efter quizzer og grupper hver gang de skal tilgås. \newline
Viewet deles op i to partial views: Èt for favoritgrupper og èt for favoritquizzer, hvor favoritgrupper er det partialview, som vises når viewet åbnes via menuknappen. Viewet med sine to partialviews ses i figur \ref{fig:Favorites}.
Quiz- og gruppenavnene agerer som links til hhv. Answer Quiz og View Group views. Favoritquizzer og -grupper kan fjernes ved at trykke på knappen i ''remove'' kolonnen.

\figur{1}{Implementering/ViewFavorites}{GUI af Favorites viewet: quizzer til venstre, grupper til højre }{fig:Favorites}

Mange forme på siden er udviklet til at kunne udvides dynamisk ved brug af AJAX og jQuery. I de controllere hvor funktionaliteten er nødvendig, som for eksempel CreateQuiz og ConfigQuiz, er der metoder som returnere den nødvendige HTML opbygning. Denne hentes med AJAX og indsættes på siden uden at hele siden behøver at loades igen.

For at udnytte \verb+ASP.NET+s evne til at arbejde på modeller redigeres navenene på alle inputfelterne så de svare til de navne der ville være hvis objekterne var oprettet i \verb+C#+.
For eksempel, hvis en quiz har ét spørgsmål og der tilføjes et nyt, får det nye input-felt, til teksten, navnet \verb+Questions[1].Text+.
På denne måde kan den samlede håndtering af form-inputtet bruge ét argument af typen \verb+Quiz+, i stedet for at skulle tage alle parametre ind én af gangen. Et udklip af koden er vist i listing \ref{lst:QuizzesJavascript}.

På linje 2-5 er AJAX-kaldet. Hvis dette går godt udføres funktionen i linje 5. Her tilføjes HTML-koden fra controlleren en \verb+div+-container. Model-binding navnet konstrueres og tilføjes alle nye \verb+input+-tags. Til sidst returneres false, for at undgå at formens almindelige submit-action udføres.

\lstsetjavascript
\begin{lstlisting}[caption=JavaScript udklip til AJAX og model-binding håndtering ved indsættelse af nye spørgsmål, label=lst:QuizzesJavascript]
$(document).on("click", "#addQuizQuestion", function() {
    $.ajax({
        url: this.href,
        cache: false,
        success: function(html) {
            // Append new question form
            $("#ConfigQuizzesQuestionsContainer").append(html);

            /* Removed code for simplicity */
			
            // Create prepend string to fix object binding
            var prePend = "Questions[" + (newCounter - 1) + "].";

            // Get all input elements and prepend input names
            $(".ConfigQuizzesQuestion:last input").each(function() {
                var original = $(this).attr("name");
                $(this).attr("name", prePend + original);
            });
            
            /* Removed code for simplicity */
        }
    });
    return false;
});
\end{lstlisting}