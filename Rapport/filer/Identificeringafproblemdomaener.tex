\chapter{Identificering af problemdomæner}

Domænemodellen på figur \ref{Domainmodel} er udviklet efter at den overordnede systemarkitektur er lagt på plads. Den fungerer som en visuel ordbog over projektet. De konceptuelle klasser er fundet ud fra aktør-diagrammet og desuden identificeret ud fra use casene. Ud fra domænemodellen kan man analysere sig nærmere ind på systemets problemdomæner.

\figur{1.0}{Domaenemodel/DomainModelFinal}{Domænemodel over det samlede system}{Domainmodel}

Ud fra ovenstående Domænemodel er der blevet identificeret følgende problemdomæner:

\begin{itemize}
	\item Webserver
	\begin{itemize}
		\item Webserver præsenterer brugergrænsefladen for bruger.
	\end{itemize}
	\item Brugergrænseflade
	\begin{itemize}
		\item Bruger interagerer med brugergrænsefladen.
	\end{itemize}
		\item Gruppe
	\begin{itemize}
		\item Bruger kan oprette en eller flere grupper. Hver gruppe kan indeholde en eller flere quizzer.
	\end{itemize}
		\item Quiz
	\begin{itemize}
		\item Bruger kan oprette en eller flere Quizzer Hver quiz har en eller flere spørgsmål.
	\end{itemize}
		\item Brugerprofil
	\begin{itemize}
		\item Bruger kan oprette sin egen brugerprofil.
	\end{itemize}
	\item Database
	\begin{itemize}
		\item Databasen gemmer quizzer, grupper og brugerprofiler.
	\end{itemize}
\end{itemize}