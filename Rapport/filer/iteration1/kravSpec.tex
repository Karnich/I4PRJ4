\section{Krav}
Første iterations produkt blev defineret som en funktionel instans af et færdigt produkt, som opfylder ASEs krav til dette projekt. Vi tænkte derfor at dette måtte være den simpleste og mest basale udgave af applikationen som muligt, og endte med at stille os selv spørgsmål om nødvendig funktionalitet.
Vi endte med at sætte en række krav til første iteration, som i overfladiske træk blev beskrevet som:
\begin{itemize}
	\item Det skal være muligt for en bruger af systemet at oprette en quiz.
	\item Quizzen skal indeholde ét spørgsmål.
	\item Spørgsmålet skal have to svarmuligheder, hvor mindst ét er korrekt.
	\item Det skal være muligt at finde en quiz ved at søge efter quizzens tags eller navn.
	\item En bruger af systemet skal kunne svare på den oprettede quiz.
\end{itemize}

Disse krav blev omformuleret til tre use cases med tilhørende aktører, som ses på figur \ref{it1-UCDiagram}

\figur{0.8}{iteration1/UC-diagram}{Identificerede aktører og deres forbindelse til UC}{it1-UCDiagram}

\subsection*{Ikke-funktionelle krav}
Ud over de funktionelle krav, der beskrives i use cases, blev der også defineret nogle ikke funktionelle krav, som kan ses i følgende afsnit. Disse krav blev til dels udarbejdet for at danne en fælles forståelse for nogle af systemets rammer.

\subsubsection*{Brugbarhed}
\begin{itemize}
	\item Programmet skal virke på Mozilla Firefox 3.5 og nyere (PC og Mac)
	\item Quiz-navnelængde er minimum 2 og maksimalt 75 karakterer
	\item Spørgsmål-navnelængde er minimum 2 og maksimalt 75 karakterer
	\item Tags-navnelængde er minimum 1 og maksimalt 75 karakterer
	\item Det er muligt at tilføje 0 til 10 tags til en Quiz

\subsubsection*{Ydeevne}
	\item Programmet skal kunne afvikles af minimum 10 samtidige brugere, som hver har 1 session åben

\end{itemize}