\section{Test}

På baggrund af de indledende arkitekturelle beslutninger omkring at anvende ASP.NET MVC frameworket, åbner muligheden for at anvende NUnit\footnote{http://www.nunit.org/ - NUnit} til at teste controllerne i systemet.

For at udnytte disse tests til fulde kombineres de med et Contenious Integration (CI)-system. ASE stiller en Jenkins-server\footnote{https://jenkins-ci.org - Jenkins CI} til rådighed for dette, hvorfor det var et oplagt valg. 
Systemet er sat op til at køre alle tests ved hvert commit i kode-basen på Git-serveren. På figur \ref{fig:iteration1:jenkins} er et udsnit af kørte tests på Jenkins-serveren for første iteration. De første sprints genererede ikke meget kode, hvorfor der ikke er mange tests der, men efter build 13 begynder der at komme en del på.

\figur{0.6}{iteration1/jenkins-tests}{Automatiskudførte tests på Jenkins-server for iteration1}{fig:iteration1:jenkins}


