\section{Grafisk brugergrænseflade design}
Brugeren interagerer med applikationen gennem brugergrænsefladen, og det er derfor vigtigt at denne indeholder simple og responsive menuer, der designmæssigt også gerne må være flotte og appelerende.  
Gennem første iteration er ens del GUI designs blevet præsenteret og diskuteret imellem projektgruppens medlemmer. Første udkast ses på figur \ref{fig:GUIScetch1}.

\figur{0.8}{iteration1/GuiD1}{Første skitse for systemets GUI}{fig:GUIScetch1}

Dette er en løs skitse, som blot blev brugt til at give gruppen en ide om i hvilken retning projektet skulle skride frem. Det blev dog hurtigt vedtaget at farvetemaet skulle være lysere, da det blev antaget at børn og unge - hvilket applikationen primært vil henvende sig til, foretrækker lysere farver.
På baggrund af disse overvejelser blev skitse nummer to, set på figur \ref{fig:GUIScetch2} udarbejdet.

\figur{0.8}{iteration1/GuiD2}{Systemets 2. GUI skitse}{fig:GUIScetch2}

Efter gennemgående overvejelser blev det dog besluttet at "åbne" designet og gøre det mindre lukket, ved bl.a. at benytte hele skærmen og udnytte skærmens top til en bjælke.

\figur{0.8}{iteration1/GuiD3}{GUI design-skitsen som arbejdes ud fra i resten af projektet}{fig:GUIScetch3}

Iterationens endelige design skitse ses på figur \ref{fig:GUIScetch3} blev udarbejdet på baggrund af disse overvejelser. En bjælke løber langs toppen, og giver brugeren en "quick-access" menu til at tilgå "create quiz" og desuden søge efter eksisterende quizzer. 
Eftersom dette blot et 1. iterations design er der stadig en del ting, der mangler. Bl.a. skal menuen til Create quiz use casen udbedres en del, så brugeren får flere muligheder og desuden opsættes på en attraktiv måde. 
