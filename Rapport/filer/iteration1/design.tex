\chapter{Design}
Brugeren integrere med applikationen gennem brugergrænsefladen og det er derfor vigtigt at denne indeholder simple og responsive menuer, der designmæssigt også gerne må være flotte og appelerende.  
Gennem første iteration er en del designs blevet diskuteret igennem. Første udkast ses nedenfor. \\

\figur{0.8}{iteration1/GuiD1}{1. skitse}{fig:ucDiagram}

Dette er en løs skitse, som blot blev brugt til at give gruppen en ide om i hvilken retning projektet skulle skride frem. Det blev dog hurtigt vedtaget at farvetemaet skulle være lysere, da det blev antaget at børn og unge - hvilket applikationen primært vil henvende sig til, fortrækker lysere farver.
Dernæst blev et mere færdigt udkast udarbejdet.\\ 

\figur{0.8}{iteration1/GuiD2}{Næste udkast}{fig:ucDiagram}


Efter gennemgående overvejelser blev det dog besluttet at "åbne" designet og gøre det mindre lukket, ved bl.a. at benytte hele skærmen og udnytte skærmens top til en bjælke. \\


\figur{0.8}{iteration1/GuiD3}{Endelige design for iteration 1}{fig:ucDiagram}

1. iterations endelige design tager disse belsutninger i brug. En bjælke løber langs toppen, og giver brugeren en "quick-access" menu til at tilgå "create quiz" og desuden søge efter eksisterende quizzer. 
Eftersom dette blot et 1. iterations design er der stadig en del ting, der mangler. Bl.a. er skal menuen til Create-quiz use casen udbedres en del, så brugeren får flere muligheder og desuden opsættes på en attraktiv måde. 
