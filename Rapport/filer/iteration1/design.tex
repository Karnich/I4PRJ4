\section{Grafisk brugergrænseflade design}
Brugeren interagerer med applikationen gennem brugergrænsefladen, og det er derfor vigtigt at denne indeholder simple og responsive menuer, der designmæssigt også gerne må være flotte og appelerende.  
Gennem første iteration er en del GUI blevet præsenteret og diskuteret i mellem projektgruppens medlemmer. Første udkast ses på figur \ref{fig:GUIScetch1}.

\figur{0.8}{iteration1/GuiD1}{Skitse 1 for systemets GUI}{fig:GUIScetch1}

Dette er en løs skitse, som blot blev brugt til at give gruppen en idé om i hvilken retning projektet skulle skride frem. Det blev dog vedtaget at farvetemaet skulle være lysere.
På baggrund af disse overvejelser blev skitse nummer to, set på figur \ref{fig:GUIScetch2} udarbejdet.

\figur{0.8}{iteration1/GuiD2}{Skitse 2 for systemets GUI}{fig:GUIScetch2}

Efter gennemgående overvejelser blev det  besluttet at ''åbne'' designet og benytte hele skærmen og udnytte skærmens top til en bjælke.

\figur{0.8}{iteration1/GuiD3}{Skitse 3 for systemets GUI. Dette er den endelige skitse.}{fig:GUIScetch3}

Iterationens endelige designskitse ses på figur \ref{fig:GUIScetch3}. En bjælke løber langs toppen, og giver brugeren en ''quick-access'' menu til til at oprette og søge efter eksisterende quizzer. Figuren viser kun et udkast af designet eftersom der skal tilføjes flere muligheder i indtastningerne mv.
