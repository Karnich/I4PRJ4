\section{Dokumentationsværktøjet Doxygen}
For at lette dokumentationsarbejdet i forbindelse med koden til applikationen, blev det valgt at anvende dokumentationsværktøjet Doxygen \citep{doxygenWeb}. Programmet kører kildekoden igennem og registrerer specielt-noterede kommentarer i koden. På baggrund af disse genereres en hjemmeside som kan bruges til at danne overblik over opbygningen og relationerne i mellem klasserne i systemet.

Doxygen fungerer på flere platforme og forskellige programmeringssprog. I dette projekt er \verb+C#+  anvendt. Med Doxygen skrives dokumentationen  i selve koden, dette muliggør at skrive sin dokumentation samtidig med at man udvikler sin kode. Ydermere bliver dokumentationen overskuelig i den forstand at dokumentationen for den pågældende klasse eller metode står sammen med den funktionelle kode og er derfor let at vedligeholde.