\section{Dokumentationsværktøjet Doxygen}
For at lette dokumentationsarbejdet i forbindelse koden tilhørende projektet, blev det valgt at anvende dokumentationsværktøjet Doxygen \citep{doxygenWeb}. Programmet køre kildekoden igennem og registrerer specielt-noterede kommentarer i koden. På baggrund af disse genereres en hjemmeside som kan bruges til at danne overblik over opbygningen og relationerne i mellem klasserne i systemet. Doxygen fungerer på flere platforme og forskellige programmeringssporg som C\#, hvilket er anvendt i dette projekt. Med Doxygen skrives dokumentationen  i selve koden, dette muliggør at skrive sin dokumentation samtidig med at man udvikler sin kode. Ydermere bliver dokumentationen overskuelig i den forstand at dokumentationen for den pågældende klasse eller metode står sammen i koden og er derfor let at vedligeholde.