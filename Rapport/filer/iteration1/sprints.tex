\section{Sprints}

I denne iteration var der tre sprints. Det første sprint var én uge og de to andre to uger.
Nøgletallene for iterationen er anført i tabel \ref{table:iteration1sprints}

\begin{table}
\centering
\begin{tabular}{|c|c|c|c|c|}
\hline 
\textbf{Sprint} 	& \textbf{Opgaver} 	& \textbf{Forventede timer} 	& \textbf{Registrerede timer} 	& \textbf{Dækning} \\ 
\hline
1 		& 16 		& 17 				& 9 						& 52 \% \\ 
\hline 
2 		& 22 		& 51 				& 32						& 62 \% \\ 
\hline 
3 		& 34 		& 83 				& 61						& 73 \% \\ 
\hline 
\end{tabular}
\caption{Tabel med nøgletal for hvert sprint i iteration 1}
\label{table:iteration1sprints}
\end{table}

Mængden af arbejde der blev taget ind steg væsentligt over de tre sprints. Vi skulle lige i gang med Scrum-tankegangen, finde ud af hvor høj en belastning vi kunne arbejde med og have en god måde at både estimere og bruge timerne på.

Logningen af timer fungerede ikke optimalt. Ud fra tallene brugte vi kortere tid på at nå i mål med sprintene, men i virkeligheden fik vi ikke logget det rigtige antal timer. På graferne på figur \ref{fig:BurndownI1S2} og \ref{fig:BurndownI1S3} vises vores time-burndown i løbet af de to sidste sprints. De viser at vores timer hovedsagligt er brugt i slutningen af hvert sprint. Det pludselige dyk til sidst er udtryk for at opgaverne ikke er blevet logget, men alligevel er løst til tiden.

\begin{minipage}{0.5\textwidth}
\figur{1}{iteration1/BurndownI1-S2}{Burndown-graf for sprint 2}{fig:BurndownI1S2}
\end{minipage}
\begin{minipage}{0.5\textwidth}
\figur{1}{iteration1/BurndownI1-S3}{Burndown-graf for sprint 3}{fig:BurndownI1S3}
\end{minipage}