\section{Database}
I denne iteration var der fokus på at få gemt brugerens oprettede quizzer, samt at kunne finde dem igen via quizzens tags, som search-funktionen vil søge på.

ASE stillede en MS SQL server \citep{sqlserverWeb} til rådighed, som blev brugt som løsning til database problematikken. Denne databaseserver er placeret på et internt netværk på ASE hvilket kræver at man er logget på ASE's VPN for at få adgang til den. I programmet DDS-Lite, som blev introduceret i 4. semesterets databasekursus I4DAB, udvikledes et ER-diagram, vist på figur \ref{fig:databasedesign}, som blev benyttet til at opsætte et databaseschema.

Schemaet indholder entiteterne Quiz, Question, Answer, Tag og QuizTagRelation. QuizTagRelation entiteten, er en mange-til-mange relation som indeholder hhv. et Quiz- og TagId

\figur{0.7}{iteration1/ER_Diagram_1}{ER-diagram over databasen i 1. iteration}{fig:databasedesign}

For at overholde princippet om separation of concerns, blev der lavet et DAL og en QuizModel. DAL blev implementeret med ADO.NET-frameworket, som muliggjorde at queries blev sendt til databasen. Dette DAL åbnede forbindelsen til databasen når der var brug for det, og lukkede den igen efter operationen var blevet udført. ADO.NET blev valgt på baggrund af en introduktion i kurset I4DAB. 

QuizModel-klassen er en facade-klasse i mellem DAL og controllerene. Her ligges logik som håndtere komplekse database-forespørgsler som når en komplet quiz skal hentes med tilhørende spørgsmål og svar. Det samme gælder når quizzer skal gemmes mv.
