\subsection*{Database-teknologier}

% Indhold
Følgende teknologier er vurderet i forbindelse med persistering i databaser.

\begin{savenotes}
\begin{tabular}{|p{3cm}|p{5.5cm}|p{5.5cm}|}
\hline 
\textbf{Navn} & \textbf{Pros} & \textbf{Cons} \\ 
\hline

Microsoft SQL
	&
	\begin{itemize}
		\item Anvendes i I4DAB
		\item Gratis server igennem Microsofts Azure\footnote{http://azure.microsoft.com/}
		\item Integreret med Entity Framework
	\end{itemize}
	&
	\begin{itemize}
		\item Licenseret brug
	\end{itemize}
	\\
	\hline

MySQL
	&
	\begin{itemize}
		\item Meget anvendt inden for webudvikling i sammenhæng med PHP
		\item Webbaseret GUI med phpMyAdmin\footnote{http://www.phpmyadmin.net/}
		\item Open-source
	\end{itemize}
	&
	\begin{itemize}
		\item Kompatibilitet med ASP.NET MVC er begrænset
		\item Integration med Entity Framework begrænset
	\end{itemize}
	\\
	\hline

Oracle Database
	&
	\begin{itemize}
		\item Integration med Entity Framework
	\end{itemize}
	&
	\begin{itemize}
	\item Licenseret brug
	\end{itemize}
	\\
	\hline
\end{tabular}
\end{savenotes} 

Ud fra ovenstående er Microsoft SQL Server valgt. Dette er begrundet af muligheden for at få adgang til en server i gennem Azure samt at ASE stiller en server til rådighed. Da undervisningen i faget I4DAB også anvender denne teknologi er det et oplagt valg.