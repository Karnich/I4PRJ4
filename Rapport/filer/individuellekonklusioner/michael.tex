%Konklusion MJ
\subsection*{Michael Toft Jensen}
Dette semesters projekt har været en del anderledes end tidligere projekter, da i tidligere projekter har der været indvolveret elektronik. På dette semester har vi været syv mand på et softwareprojekt. Det er mange personer som man skal forsøge at holde involveret i et projekt, især når vi kommer med forskellig baggrund og forskellige arbejdsmetoder, samt de fleste af os ikke har arbejdet sammen tidligere.
I starten virkede det lovende og der var bred enighed og samling om tingene, samt gruppen fik lagt en masse ideer på bordet. Jeg tror der er enighed om at vi ikke fik nået alt det vi vil nå, hvilket kan skyldes at der ikke er lagt de timer i arbejdet der skal, eller at måden vi har arbejdet på ikke har været optimal.
Udover har der som semesteret skred frem været en skæv arbejdsfordeling, og der har været en lille mangel på struktur.

Jeg har savnet at der har været en bedre koordineringen i gruppen og en bedre uddelegering af opgaver og at vi som gruppe havde siddet mere sammen og arbejdet, så man kunne få mere vidensdeling og sparring ud af hinanden. Nogle af de SCRUM elementer vi har brugt har fungeret godt, mens andre ikke er blevet anvendt som forventet og har derfor været mindre anvendelige og værdiskabende.

Jeg er blevet udfordret og har fået en masse ud af dette projekt, og lært en masse om web-udvikling både backend og frontend, og hvordan man opbygger en web applikation og hvilke udfordringer der ligger heri.