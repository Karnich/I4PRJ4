%Konklusion JC
\subsection*{Jesper Christensen}

De sidste to semesterprojekter har jeg været i en projekt gruppe kun bestående af to IKT studerende, mig selv inklusiv. Det har betydet, at vi kun var to til at lave software-arkitekturen og implementere det. Dette semester bestod projektgruppen af syv IKT-studerende, hvilket har betydet at der har været meget mere software, som skulle koordineres, og mange flere meninger samt forskellige tilgange til tingene. 

Udbyttet fra 4. semesterprojekt har været rigtig stort. Det er første gang jeg arbejder med web-teknologi, så alt inden for området har været nyt for mig, lige fra HTML til jQuery og JavaScript. Vi har haft mange teknologier og frameworks involveret i dette projekt, og de har alle været spændende at lære og arbejde med. Jeg går meget op i at udvikle mig og lære så meget jeg kan. Derfor er jeg også tilfreds med den viden og erfaring, som jeg har tilegnet mig i dette projekt.

I forhold til forrige projekter har tilgangen og arkitekturen været meget anderledes. Vi har benyttet Scrum-elementer og agil udvikling, hvilket jeg synes har været rigtig spændende. Dog har der været ting, som ikke altid fungerede for os. Arbejdsfordelingen har været skæv, da det ikke har været alle, som var lige gode til selv at have ansvaret for at tage opgaver, i stedet for at få dem givet. Derudover har vi oplevet, at det var meget svært at estimere tid på de forskellige opgaver, specielt når vi skulle bruge teknologier, som vi ikke havde arbejdet med før. Undervisningen til web-applikationer lå meget sent på semesteret, og vi har derfor skulle lære mange af tingene selv, da vi endnu ikke havde haft undervisning i det.

Jeg er meget tilfreds med det faglige udbytte jeg har fået af forløbet. Produktet synes jeg ikke helt har ramt det ambitions-niveau vi havde sat fra starten og det er selvfølgelig ærgerligt. Jeg synes at størstedelen af projektgruppen har arbejdet meget og målrettet for at opnå den funktionalitet vi har. Der er dog plads til flere forbedringer og fremtidigt arbejde.