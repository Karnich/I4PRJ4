%Konklusion LHH
\subsection*{Lars Harup Holm}
Dette semesterprojekt er det første af slagsen hvor IKT/E/EP er delt op. Det har været et spændende overgang, hvor hele projektgennemførslen er blevet omstruktureret. Dette har dog heller ikke været uden sine udfordringer. Vi har i gruppen været 7 mand, hvilket nok har været 2-3 for mange til at opnå den helt rigtige gruppesynergi, når alle 7 medlemmer arbejder med software.

Valget på at arbjde med web-udvikling har været en hård mundfuld for mig, da jeg ingen tidligere erfaring har haft med web, og  GUI-kursets forelæsning om emnet lå sidst på semesteret. Jeg synes dog, trods en langsom opstart, at mit udbytte af arbejdet er enormt, hvilket jeg til dels kan takke enkelte gruppemedlemmers overskud til at dele deres forudliggende viden, når jeg har haft problemer.

Arbejdsprocessen som en iterativ udvikling har været en anderledes, men interessant omveltning. Denne kombineret med elementer fra SCRUMs agile tilgange har været en særdeles lærerig proces både på godt og ondt. Vi har igennem projektet ændret tilgangen til SCRUMs taskboard og sprint elementer, så disse blev tilrettelagt bedre til gruppens arbejdsform. Desværre  har projektarbejdet døjet med deadlines som er overskredet og en arbejdsindsats, som for nogens tilfælde ikke er noget, der bør prales med. 

Alt i alt har jeg dog flyttet mig meget, både fagligt og personligt, i løbet af projektet. Jeg kan tage de gode og dårlige oplevelser med som erfaring om både softwareudvikling men i sandhed også om arbejdsprocesser.