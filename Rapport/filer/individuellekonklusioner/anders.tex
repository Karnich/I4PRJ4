%Konklusion ABM
\subsection*{Anders Bæk Møller}


Arbejdet med dette semesterprojekt har været enormt lærerigt, og det har både givet en bred faglig viden, men også en rigtig god erfaring med, hvordan man strukturerer arbejdet på et hold og arbejder mest optimalt for at nå sine mål. Mine primære ansvarsområder i projektet har været use casene Answer Quiz og Search, og arbejdet med dette har givet mig en god forståelse for hvordan web applikationer skal opbygges, og helt konkret har jeg fået indlæring i brugen af HTML, CSS, javascript, Jqueury-biblioteket og Ajax-teknologien. 

På det arbejdsmæssige plan har jeg opnået en masse brugbar erfaring til senere projekter: 4. semesterprojekt har budt på det første rene software projekt, og det har betydet at vi har taget de agile arbejdsprincipper i brug, hvilket har gjort at ændringer ikke har været opfattet som en belastning, men derimod som en naturlig udvikling i projektet. En anden erfaring fra projektet - som måske for mig har været en af de vigtigste har været at lære om software arkitekture. Dette har vi haft undervisning om i timerne, men det er først når man selv prøver krafter med en bestemt arkitektur at man begynder, at kunne se fordelene og ulemperne. Vi har benyttet MVC-modellen i ASP.net og det er en arkitektur, der for alvor har vist sit værd eftersom det har fjernet koplingen mellem frontend og backend-delen og det har derfor været nemt for holdets medlemmer at opdele arbejdet således at en måske har arbejdet med viewet, en anden med bussiness logikken i en controller og en tredje har arbejdet med DAL laget og tilgangen til databasen.
Arbejdet med scrums værktøjer har også givet mig en vigtig viden i, hvordan man sætter realistiske mål, arbejder struktureret og lærer at estimere hvor lang tid en given arbejdsopgave vil kunne tage (omend dette ikke altid var lige let!).


Alt i alt har jeg personligt fået enormt meget brugbar erfaring ud af dette projekt, og det er en erfaring som jeg vil kunne bruge i mine projekter i fremtiden.
