%Konklusion ABM
\subsection*{Anders Bæk Møller}


Arbejdet med dette semesterprojekt har været enormt lærerigt. Mine primære ansvarsområder i projektet har været use casene Answer Quiz og Search, og arbejdet med dette har givet mig en god forståelse for hvordan web applikationer skal opbygges, og helt konkret har jeg fået indlæring i brugen af HTML, CSS, Javascript, JQueury-biblioteket og AJAX-teknologien. 

På det arbejdsmæssige plan har jeg opnået en masse brugbar erfaring til senere projekter: 4. semesterprojekt har budt på det første rene software projekt, og det har betydet, at vi har taget de agile arbejdsprincipper i brug, hvilket har gjort at ændringer ikke har været opfattet som en belastning, men derimod som en naturlig udvikling i projektet. En anden vigtig erfaring fra projektet er læren om software arkitekture. Vi har benyttet MVC-modellen i ASP.net og det er en arkitektur, der for alvor har vist sit værd eftersom det har fjernet koplingen mellem frontend og backend-delen.
Arbejdet med projektet har også lært mig en del, omend det på nogle punkter efter min mening ikke altid har fungeret optimalt. Vi har benyttet en del af scrums værktøjer og det har helt sikkert lært mig en del om at sætte mål for projektet og opdele arbejdet i iterationer. Vi har ikke altid været lige gode til at bedømme arbejdsbyrden i starten af et sprint, men ikke desto mindre er jeg stadig blevet bedre til at estimere tidsforbruget for en given task. Gruppens medlemmer har arbejdet meget seperat, og her kunne jeg til tider godt have ønsket at der havde været flere dage, hvor vi havde siddet og arbejdet sammen på holdet. En anden ting jeg har savnet er at hver medlem har haft et fast arbejdsområde, istedet for at medlemmer har "hoppet" rundt på tasks. Dette kunne formentlig have givet en mere konsistent rollefordeling og det havde uden tvivl været nemmere at uddelegere tasks. 
 
Ser man bort fra dette har jeg alt i fået meget brugbar erfaring ud af dette projekt, og det er en erfaring som jeg vil kunne bruge i mine projekter i fremtiden.
