%Konklusion LL
\subsection*{Loc Dai Le}

Dette semesterprojekt har været spændende og lærerigt, da det har givet mig muligheden til at anvende min teoretiske viden fra de forskellige kurser på 4. semester til noget praktisk. Derudover har det giver mig en bredere forståelse for, hvordan det er at arbejde i et større team til at udvikle en specifik IT-løsning. 
Igen har vores projektgennemførelse været baseret på Scrum-princippet. Da Scrum er en meget stor og kompliceret agil udviklingsmetode, var det lidt langhåret at anvende Scrum fuldt ud i starten. Men eftersom vi fik vores Backlog og Taskboard op at køre, blev Scrum en del af arbejdsrutinen.   
Ved brug af Scrum har vi alle været på lige fod med projektet, og vores arbejde har været struktureret og velgennemført.

Udover Scrum blev projektet også genneført ved brug af iterationer. Det er første gang jeg har beskæftiget med med iterationer når det gælder projektgennemførelse. Ved at arbejde iterativt har vi opfyldt de vigtigste krav for vores system sekventiel. 

I dette semesterprojekt har vi arbejdet med webudvikling. Vi har anvendt ASP.NET MVC-frameworket til at udvikle vores web-applikation. Som projektet skrider frem, har jeg fået en bredere viden inden for webudvikling, og de teknologier der anvendes i sammenhæng med webprogrammering. Jeg har i dette projekt beskæftiget mig med teknologier som HTML5, CSS3, JavaScript, AJAX, jQuery, håndtering af databaser med Entity Framework og andre .NET frameworks. 


I gruppen har vi alle været gode til at hjælpe hinanden, og kommunikationen mellem gruppens medlemmer har været på et tilfredsstillende niveau. Der har været episoder, hvor vi har været uenig om nogen ting, men det blev altid løst på en fornuftig måde. 