\section{Arbejdsproces}

\subsection{Udviklingsmodeller}
Projektgennemførslen er forgået indenfor agile rammer, da gruppens medlemmer fra forgående semestre har opnået erfaring med denne arbejdsmetode. Det var klart fra starten at der unægteligt ville forkomme ændringer i systemet, bl.a. Fordi en del af de værktøjer, der skulle benyttes først ville blive demonstreret i andre fag senere på semestret (bl.a. Kan Microsofts entity framework nævnes: vi lærte først om denne teknologi senere i faget DAB og det stod hurtigt klart at den burde erstatte den daværende database access metode vi benyttede.). Af denne grund var det derfor oplagt arbejde agilt i modsætning til fx en mindre flexibel form som vandfaldsmodellen således at vi ikke ville blive påvirket af uundgåelige ændringer. \\
I forlængelse af de agile rammer blev det valgt arbejde iterativt med projektet - nærmere bestemt blev Scrum valgt til arbejde med projektet i. Scrum har givet gruppen nogle faste regler at følge for at nå i mål med projektet, og det har bl.a. Betydet at arbejdet er skredet frem i ugelange sprints og hver gang en ny "milepæl" i projektet er nået er projektet rykket ind i en ny iteration. Figur \ref{scrum} viser en grov skabelon over hvordan vi har udnyttet scrums egenskaber. 

\figur{0.8}{scrum}{Skabelon, der illustrere hvordan vi har benyttet scrum}{fig:scrum}

I starten af projektet blev systemets use cases udarbejdet og disse use cases har udgjort den egentlige product backlog for projektet. Ved begyndelse af en ny iteration har gruppen vedtaget hvilke use cases, der har skullet implelementeres i det kommende sprint. Gruppen har mødtes fast en gang om ugen, men derudover har vi haft timer sammen, så der er ofte opstået mere uformelle møder hvor eventuelle problemer eller nye forslag er blevet drøftet igennem. Efter hvert endt sprint er der blevet holdt sprint review, hvor sprintet er blevet evalueret så næste sprint har kunnet gøres endnu bedre.  \\
Scrum anbefaler at man udpeger en scrum master og en product owner. Eftersom vi ikke har skulle stå til regnskab over en kunde har det ikke været relevant at have en product owner og derfor er arbjedet blevet godkendt i overenstemmelse med hele gruppen. En scrum master blev fra starten valgt og gennem projektet har det været hans rolle at opretholde en god moral i gruppen og sørge for at gruppens medlemmer har ydet deres bedste.
\\
\\

Arbejdsfordelingen er sket ud fra de enkelte gruppemedlemmers interesser, og der er ikke blevet uddelt bestemte områder til medlemmerne, så det har ofte været tilfældet at en fx har stået for database-delen i et sprint mens han har implementeret en controller eller view i et andet. Denne måde at fordele arbejdet på er ikke blevet gjort fordi det nødvendigvis var det smarteste (tværtimod havde faste arbejdsområder måske været mere logisk), men derimod ud fra et ønske fra gruppens medlemmer om at opnå bedst mulig erfaring med de forskellige teknologier som projektet har budt på.


\subsection{Møder, tidsplan, logbog og referater}

\subsection{Arbejdsfordeling}

\subsubsection{Arbejdsgrupper}

\subsubsection{Rollefordeling}
