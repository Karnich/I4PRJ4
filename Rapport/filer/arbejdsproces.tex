\section{Arbejdsproces}

\subsection{Udviklingsmodeller}
Projektgennemførelsen er foregået indenfor agile rammer, da gruppens medlemmer fra foregående semestere har opnået erfaring med denne arbejdsmetode. Det var klart fra starten at der unægteligt ville forekomme ændringer i systemet, bl.a. fordi en del af de værktøjer, der skulle benyttes først ville blive demonstreret i kurser senere på semesteret. Bl.a. kan Microsofts Entity framework \citep{entityFrameworkWeb} nævnes. Dette lærte vi først om i I4DAB senere på kurset, hvorefter det stod klart at den burde erstatte den daværende database access metode vi benyttede. Af denne grund var det derfor oplagt arbejde agilt i modsætning til f.eks. en mindre fleksibel form som vandfaldsmodellen således at vi ikke ville blive påvirket af uundgåelige ændringer.

I forlængelse af de agile rammer blev det valgt arbejde iterativt med projektet - nærmere bestemt valgte vi at tage udgangspunkt i Scrum \citep{scrumWeb}. Scrums værktøjer har givet gruppen nogle faste regler at følge for at nå i mål med projektet, og det har bl.a. betydet at arbejdet er skredet frem i ugelange sprints og hver gang en ny ''milepæl'' i projektet er nået er projektet rykket ind i en ny iteration. Figur \ref{fig:scrum} viser en grov skabelon over hvordan vi har udnyttet Scrums egenskaber. 

\figur{0.8}{scrum}{Skabelon, der illustrerer hvordan vi har benyttet scrum}{fig:scrum}

I starten af projektet blev systemets use cases udarbejdet og disse use cases har udgjort den egentlige product backlog for projektet. Ved begyndelse af en ny iteration har gruppen vedtaget hvilke use cases, der har skullet implelementeres i det kommende sprint. Gruppen har mødtes fast en gang om ugen, men derudover har vi haft timer sammen, så der er ofte opstået mere uformelle møder hvor eventuelle problemer eller nye forslag er blevet drøftet igennem. Efter hvert endt sprint er der blevet holdt sprintreview, hvor sprintet er blevet evalueret så næste sprint har kunnet gøres endnu bedre.   

Selvom en masse af Scrums værktøjer er blevet benyttet er der stadig visse elementer der bevidst er udeladt. Bl.a. har der hverken været en product owner eller scrum master.  Eftersom vi ikke har skulle stå til regnskab overfor en kunde har det ikke været relevant at have en product owner og derfor er arbjedet blevet godkendt i overenstemmelse med hele gruppen.

Igennem projektet har gruppen udviklet sig og er på visse punkter blevet bedre til at strukturere arbejdet. Gruppen har fra starten af udvist disciplin og arbejdet for at færdigøre backloggen for et bestemt sprint. I starten forgik struktureringen ved at backloggen for et nyt sprint blev udarbejdet i starten af sprintet og medlemmerne kunne herefter selv vælge de tasks som de ville udføre. Denne måde at uddele arbejdet på (eller mangel på samme!) gjorde at visse tasks ikke blev taget før sent i projektet og senere begyndte hvert sprint derfor med at gruppen sammen uddelte opgaverne. Dette havde samtidig den effekt at taskene blev bedre koordineret således at tasks indenfor et bestemt område gik til samme person. 

Arbejdsfordelingen er sket ud fra de enkelte gruppemedlemmers interesser, og der er ikke blevet uddelt bestemte områder til medlemmerne, så det har ofte været tilfældet at et medlem f.eks. har stået for database-delen i et sprint mens han har implementeret en controller eller view i et andet. Denne måde at fordele arbejdet på er ikke blevet gjort fordi det nødvendigvis var det smarteste (tværtimod havde faste arbejdsområder måske været mere logisk), men derimod ud fra et ønske fra gruppens medlemmer om at opnå bedst mulig erfaring med de forskellige teknologier som projektet har budt på.

\subsection{Møder og referater}
Så vidt det har været muligt er der blevet afholdt to faste møder om ugen i projektforløbet. Det ene møde var et ugentligt statusmøde hvor gruppemedlemmer fortalte hvad de har lavet i løbet af sprintet og hvilke udfordringer de står overfor. I forbindelse med dette møde blev der også ved hver sprintafslutning, planlagt nyt sprint så de tasks der ikke blev færdige kunne komme med videre i det nye sprint. Ligeledes har der også været forsøgt at holde møde med vejleder ugentligt når der har været problemstillinger som har været nødvendige at få afklaret eller der har været en ny version af produktet som kunne vises frem.

Der er også blevet skrevet referater ved alle møderne for at dokumentere de emner og problemer som er taget op og afklaret. Disse er vedlagt på bilags-CDen. Se afsnit \ref{chap:bilagsCD} for detaljer.

\subsection{Arbejdsfordeling}
I starten af projektforløbet blev det aftalt at man selv skulle sørge for at tage opgaver på taskboardet på Redmine, men da det ikke blev anvendt som forventet, gik vi over til at uddele opgaver ved hver sprintstart. Dette gjorde vi af flere årsager, hvoraf den ene var at vi ville forsøge at skabe en mere ligelig arbejdsfordeling af opgaver og gøre det mere overskueligt at se hvem der arbejdede på hvilke opgaver. Til de store beslutninger i opstartsfasen har gruppen siddet sammen og fået afklaret de store linjer, ellers har gruppens medlemmer arbejdet meget selvstændigt i gennem forløbet, hvor de ugentlige møder har afklaret opgaver og udfordringer og Taskboardet i Redmine har været samlingspunktet for arbejdesfordelingen.

\subsubsection{Rollefordeling}
Enkelte gruppemedlemmer har haft en overordnet rolle i løbet af projektet. Bjørn har været ansvarlig for \LaTeX\ - opsætning.
