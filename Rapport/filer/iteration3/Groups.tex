\section{Groups}
%Problemer og kort om tilføjelser 

I denne iteration blev der tilføjet en vigtig feature - nemlig oprettelsen af grupper.
En bruger kan oprette en gruppe og ligeledes joine andre grupper som er oprettet af andre brugere.

I grupper kan brugere tilføje quizzer som har relevans indenfor et bestemt emne. Det kunne f.eks. være at en bruger oprettede en gruppe, der hed ''2. verdenskrig'' og tilføjede quizzer, der havde relevans til dette emne. Under oprettelse kan brugeren tilføje sine egne quizzer. Når man søger efter quizzer og har fundet en som har relation til sin gruppe, så kan man tilføje den til gruppen vedhjælp af en boks i søgeresultaterne.

Meningen med grupper er at brugere kan samles og dele quizzer som de mener gruppens brugere kunne have interesse i.  Som nævnt kunne en gruppe indeholde quizzer specifikt til et emne, men man kunne også forstille sig en skolelærer der opretter en gruppe til 2.a og tilføjer quizzer til matematik, dansk, engelsk osv. Overordnet set er grupper altså blot et værktøj til enten at samle quizzer der har relevans for et bestemt emne eller samle brugere som quizzerne har relevans for.

\figur{1}{iteration3/Group}{Create Group fra skitse til virkelighed}{fig:groupview}

På figur \ref{fig:groupview} ses til venstre det udkast der blev lavet i starten af sprintet og til højre hvordan implementeringen af dette billede er blevet.

En af udfordringerne ved group har været at få tjekket om det valgte gruppenavn allerede var taget eller ej. Dette blev løst med model-binding og Remote Validation. Remote Validation blev ligeledes tilføjet til  usernames så der heller ikke kunne oprettes brugere med ens username.

Mere information omkring Remote Validation, og hvordan det virker kan findes i dokumentationen under kapitel 7: Implementering
