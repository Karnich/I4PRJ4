\section{Krav}

I iteration 3 blev systemet udvidet med endnu en række features - derudover blev der viderudviklet på gamle features. Nye features bringer grupper og favoritter i spil, som hhv. er en måde at organisere quizzer og gemme quizzer og grupper til hurtig adgang.
I starten af iterationen blev der opstillet en række krav til features:

\begin{itemize}
	\item Mulighed for at slette spørgsmål og svar i en eksisterende quiz.
	\item Mulighed for at kunne oprette grupper og derefter kunne ændre i egne oprettede grupper.
	\item Søgning skal vise både grupper og quizzer med de søgte tags.
	\item Mulighed for at tilføje quizzer og grupper til favoritter ved søgning.
	\item Mulighed for at fjerne favoritgrupper og -quizzer.
	\item Mulighed for at tilføje quizzer fra søgeresultater til grupper som brugeren har oprettet.
	\item Mulighed for at tilføje, ændre og slette billeder på spørgsmål i en eksisterende quiz.
	\item Mulighed for at tilføje egne quizzer til gruppe, når gruppen oprettes.
\end{itemize}

Disse krav resulterede i tre nye use cases, samt ændring i en use case. Disse use cases ses i use case diagrammet på \ref{fig:It3UCDiagram}, som er det endelige use case diagram for systemet.

\begin{itemize}
	\item UC2 Find Quiz ændres til UC2 Search (som både kan finde quizzer og grupper)
	\item Ny use case: Create group
	\item Ny use case: Config group
	\item Ny use case: Config Favorites
\end{itemize}

\figur{1}{iteration3/UCdiagram}{Det endelige use case diagram for systemet}{fig:It3UCDiagram}