\section{Users}
%Problemer og kort om tilføjelser 

Under denne iteration blev der tilføjet endnu to vigtige funktionaliteter - nemlig oprettelse af en bruger samt konfigurere en bruger. Til at håndtere brugerprofiler, har vi anvendt ASP.NET Identity framework 2.0 \citep{msdnIdentityFrameworkWeb}. Dette er et bibliotek som har mange brugbare klasser, når det gælder håndtering af brugerprofiler. Klasser som gør det muligt at oprette en ny bruger med en bestemt adgangskode, slette en bruger, opdatere en bruger, ændre adgangskoden til en bruger, sende en E-mail og/eller SMS til en bruger og meget mere. Et andet argument for at anvende frameworket, er at det selv sørger for at hashe brugerens adgangskode. På denne måde sikres dataen i databasen, så uvedkommende ikke kan få adgang til de oprindelige koder.
På figur \ref{CreateUserView} ses brugergrænsefladen for oprettelse af en bruger.

\figur{0.6}{iteration3/CreateUserView}{Brugergrænseflade for Create User viewet}{CreateUserView}

Når der oprettes en ny bruger, er der krav om at brugeren skal vælge en brugernavn og en adgangskode samt oplyse en gyldigt e-mail-adresse. Brugeren kan også udfylde sit navn, men det er ikke et krav. Brugeren kan til hver en tid omkonfigurere sin profils informationer. Dog har vi valgt ikke at give brugeren lov til at ændre brugernavnet når denne er oprettet.
