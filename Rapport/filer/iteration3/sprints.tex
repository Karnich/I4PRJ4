\section{Sprints}

I denne iteration var der to sprints. De var begge to uger.
Nøgletallene for iterationen er anført i tabel \ref{table:iteration3sprints}

\begin{table}
\centering
\begin{tabular}{|c|c|c|c|c|}
\hline 
\textbf{Sprint} 	& \textbf{Opgaver} 	& \textbf{Forventede timer} 	& \textbf{Registrerede timer} 	& \textbf{Dækning} \\ 
\hline
1 		& 47 		& 70 				& 75						& 107 \% \\ 
\hline 
2 		& 79 		& 93 				& 144					&  154 \% \\ 
\hline 
\end{tabular}
\caption{Tabel med nøgletal fra iteration 3}
\label{table:iteration3sprints}
\end{table}

De forventede timer i hvert sprint er højere end sidst. Dette skyldes, at vi gerne ville nå at have de sidste essentielle funktioner med inden projektet afsluttes. Umiddelbart er resultatet for timerne for sprint 1 gode, men igen er tallene ikke retvisende i sig selv. Sprintet havde en lignende start som de tre i iteration 2. Vi nåede at lave opgaverne til vores opstartsmøde, men fik ikke estimeret nogle timer. Burndown-grafen, på figur \ref{fig:BurndownI3S1}, er derfor tæt på intetsigende. Syv opgaver blev flyttet til næste sprint fordi de ikke nåede i mål. Grafen for sprint 2, figur \ref{fig:BurndownI3S2} er væsentligt pænere. Det viser at vi har arbejdet jævnt i løbet af de to uger og næsten er nået i mål. Timeforbruget er dog væsentligt over hvad vi havde forventet. Vi har brugt 51 timer mere end forventet på de to uger.

\begin{minipage}{0.5\textwidth}
\figur{1}{iteration3/BurndownI3-S1}{Burndown-graf for sprint 1}{fig:BurndownI3S1}
\end{minipage}
\begin{minipage}{0.5\textwidth}
\figur{1}{iteration3/BurndownI3-S2}{Burndown-graf for sprint 2}{fig:BurndownI3S2}
\end{minipage}
