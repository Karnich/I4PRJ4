\section{Test}
%Simpelt diagram fra Jenkins der viser udviklingen af antallet af tests. plus noget beskrivelse

<<<<<<< HEAD
Denne iteration bød på en del ændringer af eksisterende funktionalitet i stedet for nye funktioner. Her kommer de oprettede tests rigtig i spil, da ændringerne hele tiden kan testes for om de ødelægger noget af den funktionalitet som var implementeret i forvejen.
=======
Denne iteration bød på en del ændringer af eksisterende funktionalitet i stedet for nye funktioner. Her kommer de oprettede tests rigtigt i spil, da ændringerne hele tiden kan testes for om de ødelægger noget af den funktionalitet som var der i forvejen.
>>>>>>> origin/master

På figur \ref{fig:iteration3:jenkins} er et udklip af de udførte tests for starten af projeketet og frem til slutningen af iteration 3.

\figur{1}{iteration3/jenkins-tests}{Automatisk udførte tests på Jenkins-server for iteration 1, 2 og 3}{fig:iteration3:jenkins}

Der har været en rimelig udvikling af tests under hele projektet. Dog skulle kurven, ideelt set, gerne have været mere lineær fra start til slut.