\section{Test}
%Simpelt diagram fra Jenkins der viser udviklingen af antallet af tests. plus noget beskrivelse

Denne iteration bød på en del ændringer af eksisterende funktionalitet i stedet for nye funktioner. Her kommer de oprettede tests rigtigt i spil, da ændringerne hele tiden kan testes for om de ødelægger noget af den funktionalitet som var der i forvejen.

På figur \ref{fig:iteration3:jenkins} er et udklip af de udførte tests for starten af projeketet og frem til slutningen af iteration 3.

\figur{1}{iteration3/jenkins-tests}{Automatisk udførte tests på Jenkins-server for iteration 1, 2 og 3}{fig:iteration3:jenkins}

Der har været en rimelig udvikling af tests under hele projektet. Dog skulle kurven, ideelt set, gerne have været mere lineær fra start til slut.
Integrationstestene er også lavet i denne iteration. De er lavet ud fra Collaboration-metoden og tester afhængigheder i mellem modulerne ud fra deres logiske sammenhæng i forbindelse med den enkelte use cases.

Ved afslutningen af iterationen har vi udført de accepttests der er lavet sideløbende med use case-beskrivelserne. De er alle godkendt.