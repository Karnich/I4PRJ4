\chapter*{Resume}
Denne rapport forsøger at give et detaljeret indblik i arbejdet med vores 4. semesters projekt, som omhandler applikationen Quiz Creator. Quiz Creator er hverken mere eller mindre end hvad navnet antyder: en applikation, som gør brugeren i stand til at oprette og samtidig besvare en lang række quizzer som er blevet oprettet af andre brugere. Et sådant program har mange muligheder, bl.a. til undervisningen i gymnasiet, på universitetet eller i folkeskolen, hvor læreren kan udnytte programmet til at lave quizzer til eleverne for f.eks. at kontrollere at pensum er indlært.

Projektarbejdet begyndte i starten af februar og er sidenhen løbet parallelt med resten af studiet. Gruppen besluttede fra starten at følge Scrum-frameworket således at arbejdet blev delt op i sprints. Generelt har gruppen taget en agil tilgang til arbejdet, og således er arbejdet med Quiz Creator skredet frem i iterationer som efterhånden har tilføjet flere og flere funktionaliteter. 

Ved projektets afslutning er de fleste væsentlige funktionaliteter implementeret i applikationen som forskrevet af Use Case-diagrammet. Rapporten vil give et overblik over disse funktionaliteter og hvordan vi er nået frem til dem.
