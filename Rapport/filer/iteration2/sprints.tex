\section{Sprints}

I denne iteration var der tre sprints. Det første og det sidste var på én uge og sprint nummer to var på to uger.
Nøgletallene for iterationen er anført i tabel \ref{table:iteration2sprints}

\begin{table}
\centering
\begin{tabular}{|c|c|c|c|c|}
\hline 
\textbf{Sprint} 	& \textbf{Opgaver} 	& \textbf{Forventede timer} 	& \textbf{Registrerede timer} 	& \textbf{Dækning} \\ 
\hline
1 		& 15 		& 23 				& 15						& 65 \% \\ 
\hline 
2 		& 48 		& 80 				& 77						& 96 \% \\ 
\hline 
3 		& 29 		& 45 				& 28						& 62 \% \\ 
\hline 
\end{tabular}
\caption{Tabel med nøgletal fra iteration 2}
\label{table:iteration2sprints}
\end{table}

De forventede timer i hvert sprint er højere end sidst. Dette skyldes at vi gerne ville nå længere i denne iteration.

Vi forsøgte i denne iteration at være mere forkuseret på at få registreret de brugte timer, så vi kunne se om vores estimeringer var retvisende. Dette lykkes til dels og dog nåede vi ikke helt i mål. Et helt UC blev flyttet til næste iteration. Derfor er time-dækningen ikke helt retvisende. Vi havde forventet at hver opgave tog kortere tid, så vi havde nået det hele. Dette kan også ses på burndown-graferne på figur \ref{fig:BurndownI2S1}, \ref{fig:BurndownI2S2} og \ref{fig:BurndownI2S3}. Generelt når vi ikke alle de timer som det var forventet og arbejdet bliver skubbet til det sidst.
Specielt for sprint 3 kan man også se hvordan antallet af timer stiger i starten af sprintet. Dette skyldes at vi ikke havde estimeret alle opgavernes længde inden vi begyndte sprintet. Af samme grund er grafen for sprint 2 også meget vandret fordi vi har estimeret opgaver, løst dem og lukket dem i mens sprintet er i gang.

\begin{minipage}{0.5\textwidth}
\figur{1}{iteration2/BurndownI2-S1}{Burndown-graf for sprint 1}{fig:BurndownI2S1}
\end{minipage}
\begin{minipage}{0.5\textwidth}
\figur{1}{iteration2/BurndownI2-S2}{Burndown-graf for sprint 2}{fig:BurndownI2S2}
\end{minipage}
\figur{0.5}{iteration2/BurndownI2-S3}{Burndown-graf for sprint 3}{fig:BurndownI2S3}