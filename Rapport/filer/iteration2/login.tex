\section{Login system}

I denne iteration kom login-systemet på banen. Der blev valgt at man skulle kunne login med en email med tilhørende password, eller via Facebook.

\figur{0.8}{iteration2/login_skitse}{Login GUI skitse}{fig:loginSkitse}

Login knappen aktiverer en dropdown menu, hvor i man indtaster email og password. Som skitsen viser var det tænkt, at man brugte username og password. Dog valgte vi at ændre identiteten fra username til email, på baggrund af Facebook-login. Facebook APIet sender brugerens email, som derefter bruges til at registrere brugeren i databasen. Login-knappen er også flyttet til indersiden af søgefeltet i stedet for ydersiden. Scriptet til dropdown menuen er skrevet i JavaScript. Login-funktionerne bruger AJAX med tilhørende API controllers til at logge brugeren ind.

\figur{0.8}{iteration2/login_samlede}{Login GUI}{fig:loginSekvens}

Figur \ref{fig:loginSekvens} illustrerer hvordan login er blevet implementeret på siden, hvor \ref{fig:loginSkitse} viser det første udkast til designet. \\

Login systemet bruger .NET's Identity 2.0 frameworket til autentificering. Frameworket genererer ligeledes nogle default database-entities, som vi valgte at bruge i denne iteration for at spare tid. Når brugeren er autoriseret bliver der gemt en cookie hos clienten så den genkendes.


\figur{0.8}{iteration2/ActivityDiagramLogin}{Activity diagram over de 2 login metoder}{fig:loginDiagram}

Figur \ref{fig:loginDiagram} viser hvordan loginsekvensen foregår bag facaden. Første gang en bruger benytter ''Login med Facebook'' vil personen blive registreret i databasen. Brugeren kan derefter benytte sig af ''Account'' til at oprette et password, så han en anden gang også kan logge ind med den information i stedet for at bruge Facebook. Dog bliver brugerens username i denne proces sat til at være Facebook-kontoens email, og bør nok ændres til et unikt username.

\textbf{Udfordringer:} 
Der opstod nogle udfordringer, da vi arbejdede på en intern webserver, som er stillet til rådighed af Ingeniørhøjskolen Aarhus Universitet. 

\begin{itemize}
	\item Det første problem opstår da man forsøger at registrere den interne webservers URL på Facebook. Da webserveren har en intern IP-adresse på IHAs netværk, kan Facebooks serverer ikke få adgang til denne. Dette problem står til at blive løst i en anden iteration. Muligvis med en Azure hosted webserver.
	\item Det andet problem opstår ligeledes pga. den interne server. 10.29.10.30/QuizCreator er URL på vores hjemmeside når den er published. Ajax som kalder login controlleren tror at root er 10.29.10.30 istedet for at være 10.29.10.30/QuizCreator. Dette kan godt fixes, dog vil det kun virke på enten localhost eller den interne server. Her er løsningen igen en anden webserver.
\end{itemize}