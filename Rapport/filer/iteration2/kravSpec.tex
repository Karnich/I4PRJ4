\section{Krav}
%UC-diagram, kort prosa beskrivelse af de UC vi har taget med, Ikke funktionelle krav
Til anden iteration i projektet skulle produktet udvides, og den mest logiske udvidelse ville være at tilføje brugere til systemet. I første iteration var intet personligt: Hvis man åbner websiden og opretter en quiz tilhører den ikke nogen, men alle kan se og besvare den.
I anden iteration blev der lavet en liste over krav som gav anledning til nye use cases:
\begin{itemize}
\item Det skal være muligt at registrere sig som bruger i systemet.
\item Man skal kunne logge ind som bruger eller med sin facebook konto.
\item Ændring af bruger-data som navn, password mm. skal kunne ændres af brugeren.
\item Efter login skal man kunne administere de quizzer man har oprettet.
\end{itemize}

Derudover blev der opsat et par punkter, som medvirker ændringer i tidligere use cases:
\begin{itemize}
\item Man skal kunne uploade et billede til hvert spørgsmål i en quiz.
\item Mulighed for at tilføje flere spørgsmål til en quiz.
\item Mulighed for at tilføje flere svarmuligheder til et spørgsmål.
\end{itemize}

Disse krav medførte et opdateret use case diagram, som ses på figur \ref{fig:ucdiagramfinal}.


\figur{1.0}{iteration2/UCdiagramFinal.PNG}{Use case diagram for iteration 2}{fig:ucdiagramfinal}