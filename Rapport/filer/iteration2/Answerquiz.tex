\section{Answer Quiz}

At kunne besvare en quiz er et essentielt element og ''flaskehalsen'' for projektet. Hvis brugergrænsefladen til besvarelsen af quizzer ikke føles let og overskuelig vil QuizCreator hurtigt tabe til eventuelle konkurrenter. På figur \ref{fig:AnswerquizScetch} ses den indledende designskitse af hvordan brugergrænsefladen til use casen Answer Quiz ser ud.

\figur{1}{iteration2/Answerquizskitse}{1. skitse af Answer Quiz brugergrænsefladen.}{fig:AnswerquizScetch}

Øverst i quiz-vinduet ses progressbaren, som giver hurtigt overblik over, hvor langt brugeren er nået i quizzen. I centrum ses det respektive spørgsmål som er i gang med at blive besvaret. Af denne skitse fremgår det at svarmulighederne er ''knapper'' der trykkes på, men efter en række designdiskussioner stod det klart, at svarmulighederne i stedet skulle implementeres som radio-buttons, altså boxe som kan krydses af. Dette blev vedtaget da det så nemt ville være muligt at se, hvilket svar man havde valgt, hvis man ville gå tilbage til et foregående spørgsmål.
Answer Quiz er ikke beskrevet i 1. iteration da det var en meget simpel funktionalitet den skulle indeholde. Her skulle en quiz blot bestå af et spørgsmål som skulle kunne besvares. 
I denne iteration er der blevet fyldt lidt mere funktionalitet på, og bl.a. er der benyttet AJAX til at sørge for at blot spørgsmåls-delen af siden reloades når man går til et nyt spørgsmål. 
