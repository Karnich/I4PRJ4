\section{Config og Create Quiz}
Før vi gik i gang med at implementere use casen "Config Quiz" og "Create Quiz", havde vi gjort det klar hvordan den grafiske brugergræsneflade skulle se ud for de to. Opbygning skulle nemlig være identisk for de to, da de begge behandler en quiz. På \ref{fig:CreateQuizSkitse} ses en skitse for hvordan vi ville have GUI til at se ud, og på \ref{fig:CreateQuizEndeligt} ses den grafiske brugerflade for Create Quiz. Der ses at i den nuværende iteration ikke er blevet tilføjet mulighed for at bladre gennem spørgsmålene, men blot får den listet ud efter hindanden.

\begin{minipage}{0.45\textwidth}
\figur{0.9}{iteration2/CreateQuizSkitse.PNG}{Create og Config Quiz skitse}{fig:CreateQuizSkitse}
\end{minipage}
\begin{minipage}{0.55\textwidth}
\figur{0.9}{iteration2/CreateQuizEndeligt.PNG}{Create Quiz grafisk opbygning}{fig:CreateQuizEndeligt}
\end{minipage}

\vspace{6 mm}

Vi havde brug for at udvide funktionaliteten sådan at en bruger dynamisk kunne tilføje flere spørgsmål og svarmuligheder.dd

HER TILFØJES NOGET BESKRIVELSE OM UDVIKLINGEN AF DET SCRIPT VI BRUGER TIL AT TILFØJE ANSWERS OG QUESTIONS