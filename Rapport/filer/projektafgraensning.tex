\chapter{Projektafgrænsning}
Fra starten af projektet lagde vi nogle klare afgrænsninger for hvor meget vi gerne ville have implementeret i webapplikationen. Det var vigtigt at få lagt nogle realistiske mål, da tiden til projektet er begrænset til fire måneder og 5 ECTS point. Da der er fokus på at arbejde iterativt i dette semesterprojekt, stod det dog klart at disse afgrænsninger kunne ændre sig i takt med projektets forløb. Dog blev vi enige om, at produktet skulle være en webapplikation, som udvikles på baggrund af at blive tilgået fra en stor monitor (typisk PC), og kun hvis der var tid tilovers, ville fokus være at skabe et design som også kunne tilgås fra mobile enheder - dette har dog ikke været aktuelt.

Kernefunktionaliteten i en quiz-applikation som denne, må vel betragtes at være besvarelsen af en given quiz. Kravspecifikationen og use casene blev udviklet på baggrund af denne tankegang, og det blev derfor vedtaget at sætte fokus på følgende features:

\begin{itemize}
\item At skabe en brugergrænseflade som er intuitiv at manøvrere igennem.
\item Kunne oprette en bruger og logge ind/ud.
\item Kunne oprette en quiz.
\item Kunne ændre i denne quiz på et senere tidspunkt, f.eks. tilføje flere spørgsmål.
\item Kunne oprette en gruppe og tilføje quizzer til denne.
\end{itemize}

