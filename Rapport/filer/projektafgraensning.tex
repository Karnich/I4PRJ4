\chapter{Projektafgrænsning}

Fra starten af projektet blev der lagt nogle klare afgrænsninger for projektet, så det blev overskueligt og realistisk at nå indenfor de rammer vi er blevet givet for projektet. Eftersom der "kun" er sat fire måneder af til gennemførslen var det derfor vigtigt at afgrænse projektet til nogle realistiske mål. \\
Kernefunktionaliteten i en quiz-applikation som denne må vel betragtes at være besvarelsen af en given quiz og kravsspecifikationen og use casene blev udviklet på baggrund af denne tankegang. Det blev derfor vedtaget at sætte fokus på følgende features:

\begin{itemize}
\item At kunne navigere flydende og problemfrit gennem en quiz
\item Kunne oprette en bruger og logge ind/ud
\item Kunne oprette en quiz
\item Kunne oprette en gruppe og tilføje quizzer til denne
\end{itemize}

At kunne arrangere en quiz i grupper er ikke direkte en feature, der har indflydelse på selve aspektet ved at besvare en quiz, men det bidrager til at programmet gør det lettere for brugere at finde og arrangere quizzer indenfor emner der har relevans for givne brugere.\\


Set ud fra et mere teknisk synspunkt blev der også lagt fokus på en række elementer i projektet. Bl.a. Blev det vedtaget at  applikationen i dette projekt kun skulle bestå af en webapplikation som ligger på en server og som afvikles i en browser på brugerens computer. På længere sigt vil det dog nok være relevant at udvikle en version til afspilning på en brugers smartphone eller tablet.
