\section{Erfaringer}
Arbejdet med dette semesterprojekt har givet en masse erfaring til gruppens medlemmer. Der er både opnået erfaring med rent tekniske emner, men der er også opnået bedre forståelse for fx arbejdsmetoder som scrum og strukturering af et projekt i dette omfang.\\
Teknisk er der opnået god forståelse indenfor en lang række af emner. Først og fremmest er der opnået en god forståelse for, hvordan man inkorporerer og benytter en MSSQL database i sit projekt og ligeledes hvordan man udfører CRUD operationer på denne, både ved brug af ADO.net og det mere gnidningsfri Microsoft entity framework. Desuden har gruppen fået erfaring i at udvikle web-applikationer med ASP.net frameworket som både har givet erfaring med client- men også serverside programmering. Eftersom det er en web-applikation har det ikke kunne undgåes ikke at få en bedre forståelse for markup sproget HTML, CSS og programmeringssproget JavaScript. Dette har også gjort at vi er blevet introduceret til jQuery-biblioteket og desuden også asynkrone request-kald til serveren med Ajax-teknologien.\\
MVC strukturen i projektet har givet en god forståelse for denne softwarearkitektur, og det har især vist sit værd når opgaver skulle uddelegeres, hvor en person kunne stå for design af et view, mens en anden tog sig af controlleren og en tredje arbejede med datatilgangen i DAL laget. 
\\
Dette semesterprojekt er det første af vores semesterprojekter som har budt på et rent softwareprojekt, og det har været et krav at arbjedet har skullet foregå agilt. Gruppen valgte derfor at udnytte en del af værktøjerne fra scrum. Det viste sig hurtigt at dette satte nogle gode stabile rammer for projektet: sprints har som regel taget i omegnen af 1-2 uger og det har gjort at vi har haft en konsistent arbejdsrytme og har sat nogle naturlige mållinjer for hvornår bestemte features har skulle være overstået. Desuden har vi fået fornuftig indlæring i, hvordan man for fulde udnytter et agilt arbejdsmiljø. Igennem iterationer er eventuelle ændringer og rettelser ivrigt blevet diskuteret og hvis noget har skulle refaktureres har det sjældent været noget større problem fordi arbejdet har været så iterations-fokuseret - som eksempel kan nævnes refakturering af DAL-laget: i starten blev databasen tilgået med ADO.net teknologien men efter vi i DAB-kurset blev introduceret til Microsofts Entity Framework blev dette hurtigt en erstatning for ADO.net. Sådanne relativt store ændringer i systemet ville unægteligt have været mere besværligt i en arbejdsmetode som vandfaldsmodellen. \\

På et mere formelt men bestemt ligeså brugbart plan har vi fået en masse erfaring med at arbejde på samme projekt i et større team med versionsstyringsværktøjet Git. Git har gjort det muligt for os at arbejde på samme Visual Studio projekt ved at merge ændrede filer sammen. \\

Alt i alt har arbejdet med dette semesterprojekt givet en god erfaring med en masse forskellige aspekter, og det har gået fornuftigt hånd i hånd med de resterende 4. semesterskurser og det har suppleret hinanden begge veje således at det vi har lært i undervisningen har kunnet bruges i projektet og omvendt. Erfaringerne som er blevet gjort vil uden tvivl gavne samtlige medlemmer i fremtiden, hvad end det er under bacheloren eller på arbejdspladsen.








