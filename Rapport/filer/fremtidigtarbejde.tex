\chapter{Fremtidigt arbejde}

Arbejdet med dette semesterprojekt er færdiggjort, men Quiz Creator kan udvikle sig meget mere. Mange detaljer kan tilføjes, her kan der bl.a. nævnes:

\textbf{Private indstillinger}\\
Hvis en gruppe eller quiz kun er tiltænkt bestemte brugere.

\textbf{Konfiguration af quizzer og grupper}\\
Information om quizzens formål, begrænsning af antallet af gange en bruger kan besvare en quiz, tidsbegrænsning på quizzen eller de enkelte spørgsmål.

\textbf{Præsentation af resultater}\\
Præsentation af resultater undervejs eller slet ikke hvis det er en eksamensopgave.

Selvom gruppen generelt er tilfreds med det grafiske indtryk af GUI-designet, så må det nok erkendes at det ikke er her de fleste kræfter er blevet lagt, og det vil derfor også være en oplagt mulighed at justere og forbedre designet i kommende versioner.

Dette er alt sammen smårettelser som samlet er med til at give brugeren den bedst-mulige oplevelse, men den om nok vigtigste feature som med rette kunne implementeres i en kommende version er muligheden for at afvikle applikationen på mobile-enheder såsom iOS- og Android-enheder. Dette vil passe godt med det overordnede formål med Quiz Creator og det ville betyde at brugerne ville kunne besvare quizzer i hjemmet, på arbejdspladsen, i skolen, men nu også på farten.

Hvorom alt er, så er Quiz Creator i sit nuværende stadie en funktionel og brugervenlig applikation, men der er ingen tvivl om at en masse features kan inkluderes endnu.