\chapter{Fremtidigt arbejde}

Arbejdet med dette semesterprojekt er færdiggjort, men QuizCreator kan udvikle sig meget endnu. Mange detaljer kan tilføjes: her kan bl.a. Nævnes tilføjelse af private indstillinger (fx hvis en gruppe/quiz kun er tiltænkt bestemte brugere), og generelt bør hver quiz og gruppe kunne konfigureres en hel del mere; fx bør en bruger, der opretter en quiz kunne specificere formålet med quizzen og måske indstille at en given bruger kun kan tage denne quiz en enkelt gang (hvis fx det er en eksamensquiz), og ligeledes bør denne bruger kunne indstille hvorledes quizzen er tidsindstillet, og hvordan resultatet af besvarelsen skal præsenteres til brugeren efter quizzen er taget (hvis resultaterne da overhovedet skal præsenteres). Selvom gruppen generelt er tilfreds med det grafiske indtryk af GUI-designet, så må det nok erkendes at det ikke er her de fleste kræfter er blevet lagt, og det vil derfor også være en oplagt mulighed at justere og forbedre designet i kommende versioner. \\
Dette er alt sammen smårettelser som samlet er med til at give brugeren den bedst-mulige oplevelse, men den om nok vigtigste feature som med rette kunne implementeres i en kommende version er muligheden for at afvikle applikationen på mobile-enheder såsom iOS- og Android devices. Dette ville være en oplagt feature som vil passe godt med det overordnede formål med Quiz creator og det ville betyde at brugerne ville kunne besvare quizzer i hjemmet, på arbejdspladen, i skolen, men nu også på farten. \\
Hvorom alt er så er Quiz Creator i sit nuværende stadie en funktionel og brugervenlig applikation, men der er ingen tvivl om at en masse features kan inkluderes endnu.
