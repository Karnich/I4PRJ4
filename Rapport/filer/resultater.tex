\section{Resultater}
Dette afsnit redegører for det endelige resultat af projektet efter arbejdet med det er endt. Projektets endelige resultat vurderes på baggrund af de accepttests som er blevet opstillet både for use cases og ikke-funktionelle krav. Efter accepttestene er foretaget står det klart at Quiz Creator opfylder de fleste krav som er sat til projeket. Dette afsnit vil give et kort indblik i hvordan scenariet for use casen Answerquiz fungerer. Når en bruger besøger Quiz Creator vil man blive mødt af en velkomst skærm, som er vist på figur \ref{fig:QuizCreateView}

\figur{0.6}{resultat/forside}{Forside på Quiz Creator}{fig:QuizCreateView}

Herfra kan brugeren vælge at søge efter en quiz, logge ind og hvis man er logget ind kan man desuden oprette en quiz, se sine grupper eller quizzer. Hvis brugeren vælger at søge efter en quiz vil de quizzer der matcher søgeordene (dette gælder både tags som er tildelt quizzen og quizzens navn), blive vist for brugerne. Quizzerne opfylder de ikke-funktionelle krav og således vil en quiz mindst indeholde 2 spørgsmål og højst 75, ydermere vil en quiz bestå af mindst 1 tag og højst 75, når en quiz vælges vil brugeren blive præsenteret for det første spørgsmål, og man har nu mulighed for at besvare dette og dernæst gå videre til næste spørgsmål eller gå til et vilkårligt spørgsmål i quizzen via quickaccess baren over spørgsmålet. Når alle spørgsmål er besvaret vil man få mulighed for at trykke på en knap for at afslutte quizzen. Når man gør dette vises en skærm, der afslørere resultaterne for den netop besvarede quiz.

Alt i alt er projektet implementeret tilfredstillende i forhold til gruppens ambitioner. 

