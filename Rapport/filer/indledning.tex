\chapter{Indledning}
Quizzer bruges i dagligdagen til alverdens ting. De benyttes i alt lige fra Trivial Pursuit til seriøse eksamener på gymnasiet eller på universitetet. Quizzer er alsidige - de kan være så korte eller lange som man ønsker, de kan indeholde spørgsmål om alt lige fra hvilken dag Danmark blev invaderet i 2. verdenskrig til hvilken mel-type der er bedst at bruge i fastelavnsboller (*hvedemel er svaret).

Quizzer befinder sig overalt i dagligdagen og dog er der stadig ikke blevet skabt en solid platform, hvor brugere kan oprette og dele quizzer med hinanden. Med Quiz Creator vil vi tage de første spæde skridt i retning mod denne vision: Projektet skal i den første version opfylde en række use cases, herunder bl.a. Oprettelse af quiz, besvarelse af quiz, søgning af quiz, oprettelse af bruger, log in, log ud, oprette grupper til at sortere quizzer indenfor et bestemt område osv. En applikation af denne type er fleksibel nok til ikke kun at høre hjemme på én platform, og således vil det give mening både at lave en version til PCer, Smartphones og tablets. Grundet den begrænsede tid vi har til projektet vil vi dog kun fokusere på PC-versionen som skal være en internet-applikation. 

Denne rapport beskriver udviklingsforløbet igennem projektet. Eftersom der er blevet arbejdet agilt med Scrum i sprints, har vi valgt at opdele rapporten i iterationer således at læseren får nemt overblik over hvordan arbejdet er skredet frem, og hvilke milepæle der er sat hvornår.